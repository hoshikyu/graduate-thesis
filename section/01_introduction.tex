\chapter{序論}
\label{chap:introduction}

\section{本論文の背景}
従来行われている麻雀のAIの研究では、機械学習や探索などと言った従来のゲームAIの一般的な手法が取られている。
麻雀の戦略を適用するというよりは、AIの手法を用いて最適なモデルを新たに制作するという考え方である。

一方麻雀の戦略自体は、多数の試合の統計的な観点、シミュレーションやモデルなどの観点から、既に多くの研究がされている。

麻雀というゲームは不完全情報ゲームであるがために、様々な要素が入り組んで複雑な要因がたくさんあるように考えられている。しかし、実際に成績に大きく影響を与える部分はシンプルで、簡単な戦略で示すことができることが理論レベルでわかっている。

そこで筆者としては、理論レベルで言われている戦略の重要な部分を実装することで、十分に戦えるシステムとなり得るのではないかと考えた。

もちろん、最強レベルの麻雀AIを目指すのであれば、複雑な要因を比較することが重要になるだろう。しかし重要部分である戦略を適切に行うだけでも、先行研究のAIような成績が十分期待できるのではないかということである。

本研究では、麻雀の既存戦略を実装して実際に対人戦を行って成績を集計し評価する。これにより、理論レベルで言われている戦略が実際どの程度の影響を与えるものなのかということを検証する。

\section{本論文の目的}
麻雀は不完全情報ゲームであるため、特定の戦略がどの程度成績に与えるかが不明確な点が多い。
理論レベルで実証されていることも、実際にその理論を用いて実戦に挑んだ場合、どのような効果があるのかを明らかにすることが目的である。

今回実装する戦略を用いた成績が先行研究に匹敵するレベルであれば、人が麻雀を学ぶときにはそれを学ぶことが大事であることがより裏付けられる。
また、仮に明確に劣る場合であっても、今回の戦略を少なくとも行うことでどの程度の成績が期待でき、残りどの程度の差があるのかを明確に数値として示すことができる。

\section{本論文の構成}
本論文における以降の構成は次のとおりである。

\ref{chap:strategy}章では、先行研究によって麻雀において成績に大きく影響を与える部分を明らかにし、簡略化した上で今回実装する部分を定義する。
\ref{chap:implementation}章では、前章で定義した戦略を、実際に実装した麻雀自動打ちシステムについて解説する。
\ref{chap:evaluation}章では、\ref{chap:implementation}章で実装した自動打ちシステムを自身の研究同士・先行研究と比較をし、評価を行い、その結果について考察する。
\ref{chap:conclusion}章では、本論文のまとめと今後の展望についてまとめる。