\chapter{序論}
\label{chap:introduction}

\section{本論文の背景}
従来行われている麻雀のAIの研究では、基本的に機械学習的なアプローチやゲームAI的なアプローチによってモデルケースを考えるようなやり方が多い。
すなわち、麻雀の論理的戦略を考えるというよりは、ゲームAiのアルゴリズムによって原因は分からないがうまくいくモデルを作ろうというアプローチが基本的である。

しかし、麻雀においては今まで人間が体系化してきた戦略、特に科学する麻雀の統計的分析により、最も大きく成績に影響を与える要因ということがしっかりわかっている。

筆者の仮説としては、麻雀というゲームは不完全情報ゲームであるがために、様々な要素が入り組んでどれが複雑な要因がたくさんあるように考えられているが、実際に成績を影響させるものとしては、十分簡単なアルゴリズムで説明でき、戦略も簡単にできるのではないかということである。

もちろん、最強レベルの麻雀戦略を作るためには、実際には複数の複雑な原因を比較しなければならないが、もっと簡単に影響すべきところだけをとって、比較すれば、先行研究のような成績が十分期待できるのではないかということである。

本研究では、麻雀の既存戦略による、簡単なアルゴリズムに酔って、実際のところどの程度の成績が期待できるかというころを検証し、麻雀の戦略において成績影響度が高いことは一体何なのかを出来る限り突き止めていきたいと思う。

科学する麻雀の戦略によって、麻雀に与える影響と言うのは何が大きいかということは既にわかっている。それを検証していきたい。

科学する麻雀によって、例えば回し打ちは無意味など、一般的には細かい成績を十分に分けると言われている戦略でも、無視して行ったほうが実際には良い結果がでるなどの検証は既に理論レベルで構築されている。

実際はどうなのか、仮説としてはそれが正しく、十分な検証になると思われるが、実際どうなのかは実装してみて考えたいと思う。

\section{本論文が着目する課題}
麻雀というゲーム自体は不完全情報ゲームということもあり、そもそもの戦略がどの程度成績に与えるかが不明確な点が多い。
理論レベルで実証されていることも、実際にその理論を用いて実戦に挑んだ場合、どのような成績をして、どのような効果が期待できるのかを、本論文では簡単な戦略に絞り、それを平均順位などのパラメータから効果を検証していきたいと思う。

\section{本論文の構成}
本論文における以降の構成は次のとおりである。

\ref{chap:video-transmission}章では、本論文を理解するための前提となる、汎用的な映像伝送システムについての解説をする。色空間と帯域の関係などにも触れる。
\ref{chap:network-transmission}章では、本論文の要となるネットワークを活用した映像伝送システムについて、本論文での実装との違いを踏まえ紹介する。
\ref{chap:implementation}章では、ネットワークを活用した映像伝送システムを、ソフトウェアとハードウェアで設計、実装したことについて解説をする。
\ref{chap:evaluation}章では、\ref{chap:implementation}章で実装した映像伝送システムを、既存のIP伝送装置と比較をし、評価を行い、その結果について考察する。
\ref{chap:conclusion}章では、本論文のまとめと今後の展望についてまとめる。

\section{本論文の目的}
麻雀というゲーム自体は不完全情報ゲームということもあり、そもそもの戦略がどの程度成績に与えるかが不明確な点が多い。
理論レベルで実証されていることも、実際にその理論を用いて実戦に挑んだ場合、どのような成績をして、どのような効果が期待できるのかを、本論文では簡単な戦略に絞り、それを平均順位などのパラメータから効果を検証していきたいと思う。
