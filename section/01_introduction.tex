\chapter{序論}
\label{chap:introduction}
\section{本論文の背景}
近年、2人零和確定完全情報ゲームであるチェス、オセロ、将棋といったゲームではAIの研究がめざましく、人間のトッププレイヤと同等以上の成績を記録している。
一方で多人数不完全情報ゲームである麻雀では未だそれに及ばない状態である。これは従来のゲーム木の探索手法を適用するのが難しいためである。
多人数ゲームでは状況によってプレイヤの行動の目的が違うといった難しさがある。この難しさを解消するために、先行研究\cite{LinUCB_mahjong}\cite{bakuuti2013}でも多数行われている通り、相手を考慮しない 1人麻雀プレイヤに対して和了率を求める評価を採用する。麻雀を行う時の人間の思考は1 人麻雀をもとにしているため、1人麻雀から4人麻雀への拡張は容易だと考えられる。

\section{本論文が着目する課題}
1人麻雀においてある局面からゲーム木を展開しようとした場合、次にどの種類牌をツモるかはわからないため、ランダムに決定しノードを展開していくことになる。これを繰り返し行うと探索空間が大きくなりすぎるため、手の選択が難しくなる。これを対処するために、あらゆる手法が提案されている。このような場合に木を展開せずに有望な手を選択する手法として、Upper Confidence Bound (UCB) \cite{UCB}がある。UCB ではある局面から考えられる全ての手に対して, よい結果 を返しそうな手を重視しつつ、何度もプレイアウトと呼ば れるランダムシミュレーションを行うことで 最善の手を決定する手法である。
UCB は局面ごとにこの探索を実行するが, 各局面を完全 に別の局面として扱うため他の局面の探索において得た情 報を利用することができない。この問題を解決するためにLinear UCB (LinUCB) \cite{LinUCB} という手法が提案されている。これは、局面を特徴で表す ことで、対象とする局面が異なっても, それまで対象とした局面の情報を利用できるといったものである。
これらは麻雀について適用した例が報告されている\cite{LinUCB_mahjong}が、いずれの方法も平均プレイヤーに満たない成績となった。

% しかし、UCB1 と LinUCB の評価値計算において信頼度の 計算式が異なることや、特徴量の設計に問題があるなどの理由でいい結果は出ていなかった。 
本研究では、モンテカルロ法の問題点である、麻雀に適用すると探索空間が大きくなりすぎてうまく適用出来ない問題に対して、探索空間を小さくするために、2つの解決策を行う。一つはモンテカルロ法のシミュレーションを直接牌姿に当てはめるのではなく、牌姿ごとの局面で数理的に比較できる内容を数理計算を用い削減する方法である。もう一つは、三向聴以下に対してはシャンテン数を下げるように打ち、モンテカルロ法を用いる部分を限定する方法である。麻雀の牌理に関しては、上がりに近づく選択ほど重要である\cite{gendai}ため、聴牌に近い打牌選択においてモンテカルロ法を適用することで、成績に影響を大きく与える部分でモンテカルロ法の効果が発揮できることが期待される。
\section{本論文の目的}
本研究では和了率を理論値計算によって評価することと和了に近い段階へのモンテカルロ法の部分適用によって、麻雀に対してモンテカルロ法をそのままうまく適用できない問題に対して解決策を提案する。多人数ゲームでは状況によってプレイヤの行動の目的が違うといった難しさがある。この難しさを解消するために、先行研究[1,2]でも多数行われている通り、相手を考慮しない 1人麻雀プレイヤに対して和了率を求める評価を採用する。従来のモンテカルロ法を麻雀に適用した例では平均プレイヤーレベルの実力が出ていないため、本手法によってそれらを上回ることが期待される。
\section{本論文の構成}
本論文の構成について述べる。
\ref{chap:relevantstudy}章では、不完全情報ゲームの関連研究について述べる。
\ref{chap:approach}章では、1人麻雀にモンテカルロ法を適用する際に、適用範囲を限定し、比較内容を数理的和了率によって評価する手法を提案する。
\ref{chap:evaluation}章では、\ref{chap:approach}章で提案した手法で実装したAIに1人麻雀を打たせ、各種パラメータを比較する。また、4人麻雀で打った際の成績の評価も行う。
\ref{chap:conclusion}章では、本手法によって得られた知見について述べる。
