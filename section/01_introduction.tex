\chapter{序論}
\label{chap:introduction}

\section{本論文の背景}
従来行われている麻雀のAIの研究では、基本的に機械学習や従来のアルゴリズムを使ったゲームAI一般的な手法が取られている。
一方、麻雀の戦略自体を掘り下げることは、まだまだ可能である。 この不一致を解決することが大事。

麻雀においては今まで人間が体系化してきた戦略、特に科学する麻雀の統計的分析により、最も大きく成績に影響を与える要因ということがしっかりわかっている。

筆者の仮説としては、麻雀というゲームは不完全情報ゲームであるがために、様々な要素が入り組んでどれが複雑な要因がたくさんあるように考えられているが、実際に成績を影響させるものとしては、十分簡単なアルゴリズムで説明でき、戦略も簡単にできるのではないかということである。

もちろん、最強レベルの麻雀戦略を作るためには、実際には複数の複雑な原因を比較しなければならないが、もっと簡単に影響すべきところだけをとって、比較すれば、先行研究のような成績が十分期待できるのではないかということである。

本研究では、麻雀の既存戦略による、簡単なアルゴリズムによって、実際のところどの程度の成績が期待できるかというころを検証し、麻雀の戦略において成績影響度が高いことは一体何なのかを検証する。

科学する麻雀によって、例えば回し打ちは無意味など、一般的には細かい成績を十分に分けると言われている戦略でも、無視して行ったほうが実際には良い結果がでるなどの検証は既に理論レベルで構築されている。→至って単純なアルゴリズムでできるのではないのか?が仮説。


\section{本論文が着目する課題}
麻雀は不完全情報ゲームであるため、特定の戦略がどの程度成績に与えるかが不明確な点が多い。
理論レベルで実証されていることも、実際にその理論を用いて実戦に挑んだ場合、どのような効果が期待できるのかを、明らかにすることが目的である。

\section{本論文の目的}
麻雀は不完全情報ゲームであるため、特定の戦略がどの程度成績に与えるかが不明確な点が多い。
理論レベルで実証されていることも、実際にその理論を用いて実戦に挑んだ場合、どのような効果が期待できるのかを明らかにすることが目的である。

\section{本論文の構成}
本論文における以降の構成は次のとおりである。

\ref{chap:}章では、先行研究によって麻雀において成績に大きく影響を与える部分を明らかにし、簡略化した上で今回実装する部分を定義する。
\ref{chap:implementation}章では、前章で定義した戦略を、実際に実装した麻雀自動打ちシステムについて解説する。
\ref{chap:evaluation}章では、\ref{chap:implementation}章で実装した自動打ちシステムを自身の研究同士・先行研究と比較をし、評価を行い、その結果について考察する。
\ref{chap:conclusion}章では、本論文のまとめと今後の展望についてまとめる。