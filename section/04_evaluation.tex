\chapter{評価}
\label{chap:evaluation}
\section{実験概要}
\section{1人麻雀プレイヤによる和了率の比較}
提案手法で提示した、期待和了平均順目の評価によるモンテカルロ木探索を1人麻雀に適用し、性能を評価した。期待和了平均順目をn順先の変化まで評価することによって性能が変わる可能性があるので、1順、2順、3順までの変化を考慮するものをそれぞれ分けて実験した。
それぞれに対して100局のテストデータと10,000 局のテストデータを与え、あがることができた局数を計測した。テストデータとは、全ての牌をランダムに並べたものを100セット用意したデータである。このデータから 13 牌を初期牌として 与え, その後牌を引いて切る動作を 27 回行い, その中であ がれたかどうかを確認した。100 局のテストデータで人間 のプレイヤとの比較を行い, 10,000 局のテストデータで各 手法の性能の評価を行った。


\section{4人麻雀における先行研究とのレートの比較} %麻雀サーバーとの対戦
本研究では1人麻雀における和了率の向上を目指すため、和了率の数理的評価とモンテカルロ法を適用する部分を限定する手法をとった。佐藤らは、1人麻雀における和了率を有効牌を数え上げて大きくなるようにすることで、和了率の最大化を図り、これを4人麻雀で打たせレートを取った。同じように本研究手法でも4人麻雀で打たせた結果を比較した。
\\
表
\\
このように、レートは本手法の方が上回る結果となった。ここで保証安定レートについて比較をする。

