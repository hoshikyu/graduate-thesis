\chapter{評価}
\label{chap:conclusion}

評価方法としては、実際に天鳳一般卓で以上に述べた戦略を実装した自動うちシステムを実際に対戦させ、成績を取る。
その際取るパラメータとして、「平均順位」「和了率」「放銃率」などのパラメーターが基本的である。


\section{自作アルゴリズム同士の比較}


オリの技術に関しては非常に重要であるが、実際のところオリの技術というのは成績にどう影響するのかということを、自信のベースラインとの比較によって検証したいと思う

オリを実装していない全ツmy自動打ちシステム   VS  オリを実装したmy自動打ちシステムの比較 


また、統計的検証により、信頼区間やウェルチ検定などで、どれくらいの差があるかということが証明できる。
ただし、オリを実装すれば実際のところ成績が改善することは当たり前なので、本当に自分の降りだけでいいのか?というのを比較するのは次の評価のセクションで考える。

\section{関連研究との比較}

評価方法としては、爆打が同じように天鳳一般卓で打った成績が存在するので、それに対して比較してみることがいいかもしれない。
実際のところ、爆打はあらゆる要素を比較して機械学習してモデルケースを作っているが、実際それがどの程度の影響を及ぼしているのか?を比較したい。

爆打にも、 ベースライン と 提案手法 が存在するので、それをそれぞれ比較することになるかも。これについてはもう少し考えないといけない。

仮説では、自分の簡単な戦略でも十分に平均的なプレイヤとしての成績が期待できるのでそれが実際どの程度の差があるか、各要素を比較することはどれだけ重要なのか、を数値的に明らかにしたいと考えている。



