\chapter{映像伝送システム}
\label{chap:video-transmission}
\section{ビデオカメラ}
\section{ディスプレイ}
\section{インターレース}
\section{色空間}

映像では一般的に、RGBやYUV, YCbCr, YPbPrといった色空間で表現される。
色差成分を間引くことにより、

\section{帯域}

帯域は次のようになる。

表は次のように出力される(表\ref{tb:video-bandwidth})。

\begin{table}[htbp]
  \caption{表の例}
  \label{tb:video-bandwidth}
  \begin{center}
  \begin{tabular}{l|c|r|l|l}
    \hline
    解像度&フレームレート&色空間&ピクセルあたりのビット数&帯域\\\hline\hline
    3840x2160&60P&RGB&X bit&XX Gbps\\\hline
    3840x2160&30P&RGB&X bit&XX Gbps\\\hline
    3840x2160&30P&RGB&X bit&XX Gbps\\\hline
    3840x2160&30P&YUV 444&X bit&XX Gbps\\\hline
    3840x2160&30P&YUV 422&X bit&XX Gbps\\\hline
    3840x2160&30P&YUV 420&X bit&XX Gbps\\\hline
    1920x1080&60P&RGB&X bit&XX Gbps\\\hline
    1920x1080&60P&YUV&X bit&XX Gbps\\\hline
    1920x1080&30P&RGB&X bit&XX Gbps\\\hline
  \end{tabular}\end{center}
\end{table}


\section{インターフェース}
\subsection{HDMI 1.4/2.0}
\section{伝送手法}
\section{まとめ}

\chapter{ネットワークを活用した映像伝送}
\label{chap:network-transmission}
\section{仮説}
\subsection{Ethernetを活用するメリット}
\section{目的}
\section{構成}
\section{関連研究}

\chapter{システムの設計・実装}
\label{chap:implementation}
\section{UoIP} % UHD over IP
\section{システム構成}
\section{ソフトウェアによる実装}
\section{ハードウェアによる実装}
\subsection{FPGAの回路設計}

\chapter{評価}
\label{chap:evaluation}
\section{評価手法}
\section{計測}
\subsection{トラフィック}
\subsection{遅延}
\subsection{重量}
\section{考察}

\chapter{結論}
\label{chap:conclusion}
\section{本研究のまとめ}
\section{今後の課題と展望}

citation\cite{hoge09}
