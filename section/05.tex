\chapter{結論}
\label{chap:conclusion}

以上のような評価によってわかることを書く。

今まで麻雀研究によって研究されてきた理論として、成績に大きく影響を与える簡単な要素にしぼり、実際に自動打ちシステムを使い実戦を行ってみた。結果、次のようなことがわかった。

\section{攻める戦略について}

攻める戦略については和了率を比較することに酔って 〜〜 ということがわかった。
やはり仮説どおり〜〜である。

\section{降りる戦略について}

オリの戦略については、意外にも〜〜〜であることがわかった。
すなわち、今まで言われていたように細かいベタオリの戦略は〜〜であるということである。

そして数値としての違いとして〜〜〜であるから、〜〜ほどの差が存在するということである。

\section{本研究のまとめ}

筆者の仮説の通り、従来の麻雀戦略の研究により、麻雀というゲームにおける成績というのは、簡単な条件によって十分うまく打てるものであり、細かい比較についてはそれ以上のレベルをこなす必要があるときだけであるということがわかった。
したがって、麻雀AIに関しても、基礎戦略についてはまずは条件分で書くなどのことが、結果的に最強戦略になる可能性があるなどが考えられるのである。


\section{今後の課題と展望}

今回の研究では〜〜ということがわかったが、いろいろな点棒が考えられる
・麻雀AIとしての最強なものを作るには、今回の研究を使って実際どうすればよいのか
・人間に麻雀を教える際には、以上のような簡単な戦略から教えることで、十分に平均的プレイやを超えることが出来、難しい戦略を取り入れる必要が無いということがわかった
後者については、大きな進歩である。
不完全情報ゲームにおける問題点としては、人間戦略を勉強する際にその優先度がわかりにくいということであったが、実際にこれでわかるようになったのである。

citation\cite{hoge09}




