% ■ アブストラクトの出力 ■
%	◆書式:
%		begin{jabstract}〜end{jabstract}	:日本語のアブストラクト
%		begin{eabstract}〜end{eabstract}	:英語のアブストラクト
%		※ 不要ならばコマンドごと消せば出力されない。

% 日本語のアブストラクト
\begin{jabstract}

ゲームAIによる昨今の成果として、囲碁や将棋などのゲームでは、人間のチャンピオンレベルの実力に至っている。一方、麻雀においては未だそれらに及んでいない。麻雀は多人数不完全情報ゲームに分類され、多人数性による複雑性と不完全情報による不確定性を理由に、適切なモデルを作成することが難しいからである。多人数性の問題では、目的の異なる相手プレイヤーの行動を予測することが難しい。また、不完全情報の問題では、未知の情報を仮定することが必要となるため、期待値による行動の決定を行わなければならない。実現可能なパターン全てを探索して比較することも、探索空間が膨大になりすぎるため非現実的である。

本研究では、以上に述べたような麻雀の性質から、問題を和了に限定し、和了率を高めるアルゴリズムに注目する。和了率は麻雀において得点する頻度を表す指標であり、これを最大化することは成績を上げるために重要である。和了率を高めるための打牌アルゴリズムとして、本研究では期待和了巡目の評価を利用する手法を提案する。期待和了巡目とは、与えられた牌姿において和了までにかかる平均消費巡目を理論値的に概算して求めたものである。これが最も小さくなるような打牌を選択することで、和了率の最大化を目指す。先行研究では単に有効牌の数だけで和了のしやすさを比較していたため、有効牌同士の優劣を精密に比較することができない問題があった。本研究の期待和了巡目を用いた手法では、有効牌をさらにブロックと呼ばれる種類ごとに分類することで、この問題を解決し、和了率を上げることを目的とする。

この手法の効果を確かめる実験として、多人数性を取り除いた1人麻雀の成績と、通常の4人麻雀の成績を、シャンテン数や有効牌を指標としたアルゴリズムや人間プレイヤーと比較した。本手法で構築したアルゴリズムは1人麻雀において、シャンテン数だけで比較するアルゴリズムよりも約10%、有効牌の数だけで比較するアルゴリズムよりも約3%高い和了率を示すことがわかった。4人麻雀では、それぞれよりわずかに和了率とレーティングが上回ったが、統計的に優位な差ではなかった。


\end{jabstract}


% 英語のアブストラクト
\begin{eabstract}

Recent achievements by game AI have led to human champion's ability in games such as Go and Shogi. On the other hand, Mahjong has not yet reached it. Mahjong is categorized as a multi-player incomplete information game, and it is difficult to create an appropriate model for reasons of complexity due to multiplicity and uncertainty due to incomplete information.

In this thesis, we propose a method to use the evaluation of expected winning cycle as batting tile algorithm to raise the rate of winning.

As an experiment to confirm the effectiveness of this method, we compare the performance of one mahjong who removed the multiplayer and the result of ordinary four person mahjong with the algorithm of related research and human player.

It was found that the algorithm constructed by this method shows higher winning rate than the algorithm to compare by merely the number of effective tiles in one person mahjong. In the 4-person mahjong, the rate of winning and ratings were slightly higher, but it was not statistically significant difference.

\end{eabstract}
