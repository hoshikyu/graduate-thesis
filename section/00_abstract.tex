% ■ アブストラクトの出力 ■
%	◆書式:
%		begin{jabstract}〜end{jabstract}	:日本語のアブストラクト
%		begin{eabstract}〜end{eabstract}	:英語のアブストラクト
%		※ 不要ならばコマンドごと消せば出力されない。

% 日本語のアブストラクト
\begin{jabstract}

近年、2人零和確定完全情報ゲームであるチェス、オセロ、将棋といったゲームでは人間のトッププレイヤと同等かそれ以上の実力を持つコンピュータプレイヤが提案され
ている。一方で多人数不完全情報ゲームである麻雀では未だそれに及ばない状態である。これは従来のゲーム木の探索手法を適用するのが難しいためである。
本論文では、探索空間が大きくなることでうまく適用できなかったモンテカルロ法について、期待和了平均順目の数理的な評価を利用することで探索空間を小さくし、適用する手法を提案する。この手法の効果を確かめる実験として、多人数性を排除した1人麻雀における和了率の比較を、人間のプレイヤや先行研究AIと行った。またそれをそのまま4人麻雀で対戦させた場合の比較も行う。解析の結果、本手法で構築したAIは1人麻雀において、平均レベルのプレイヤの和了率を出すことができることがわかった。また、4人麻雀に適用した場合、シャンテン数を下げるように打つAIに比べて高いレートを出すことが可能であることがわかった。


\end{jabstract}


% 英語のアブストラクト
\begin{eabstract}

In recent years, computer games that are equal to or better than the top players of humans have been proposed for games such as chess, Othello, and shogi, which are two-person zero settled perfect information games
ing. On the other hand, Mahjong which is a multiplayer incomplete information game has not yet reached it. This is because it is difficult to apply conventional search methods of game trees.
In this paper, we propose a method to apply by restricting mathematical evaluation and applied parts of the Monte Carlo method which could not be applied successfully by increasing the search space.
For evaluation, comparison of the rate of winning in one mahjong excluding multiplayer was done with human players and previous studies AI. We also make comparisons when we played it with 4 mahjong as it is.
As a result of the analysis, we found that the AI ​​constructed by this method has comparable ability to that of the average level player.

\end{eabstract}
