\chapter{結論}
\label{chap:conclusion}

本章では、本研究が提案する期待和了巡目を用いた打牌アルゴリズムによって得られた知見について述べる。

\section{本研究のまとめ}
本研究では、各牌姿における期待和了巡目を用いて、有効牌を更にブロックに分けて和了のしやすさ評価し、和了率を向上する打牌を選択するアルゴリズムを提案した。評価として、1人麻雀における和了率、4人麻雀における和了率・放銃率・レーティングを比較した。

本研究の提案手法では、1人麻雀の和了率が「シャンテン数を下げるように打つアルゴリズム」よりも約10%、「有効牌が多くなるように打つアルゴリズム」よりも約3%高いということが示された。また、平均プレイヤーよりも約8%高い和了率であることが確認された。一方4人麻雀では、本研究手法が有効牌を扱ったアルゴリズムより和了率とレーティングがわずかに高いことが示された。しかし、大きな有意差は無く、どちらも平均プレイヤーに及ばなかった。

\section{本研究の結論}
1人麻雀における和了率の結果から、本研究の期待和了巡目を扱った打牌アルゴリズムは、和了率を上げる観点で有用であることがわかった。麻雀において和了率が高いことは成績において重要であるため、その観点から麻雀の成績を上げることに利用できる可能性が高いということがわかった。しかし、4人麻雀においての和了率やレーティングの結果から、多人数性が存在する鳴きやオリの観点の考慮を行わないとこれ以上の成績向上が難しいことも確認できた。これは、1人麻雀の和了率が平均プレイヤーに統計的に明らかに優っているアルゴリズムでも、4人麻雀においては平均プレイヤーに及ばない例が存在するからである。

% 本稿では, 牌譜の局面からの教師あり学習や異なる探索 の結果の共有ができる LinUCB を 1 人麻雀に適用すること を提案した. LinUCB と, 比較手法として UCB1 と事前学 習を利用する手法を 1 人麻雀に適用し, あがった局数を比 較することで評価を行った. その結果いずれの手法も人間 のプレイヤには及ばず, 3 つの手法では UCB1 が最も良い あがり率を出す結果となった. しかし, UCB1 と事前学習 のみを用いる手法の性能が近かったことから, 事前学習が 不完全情報ゲームに対してある程度有効であるということ が分かった. また, Plain LinUCB, 事前学習+LinUCB の 結果から特徴量の設計が不適切であったことが考えられる.
% 今後は特徴量の見直して実験を行い, LinUCB の性能評 価を再度行いたい. 本実験では, LinUCB と事前学習を利用 する手法を分けて評価を行ったが, LinUCB で成果が出る ようになれば, これらを組み合わせることで更に性能が向 上することが期待されるのでこれについても評価を行いた い. また, UCB をモンテカルロ木探索に適用した UCT [8] が成果をあげていることから LinUCB をモンテカルロ木探 索へ適用することも検討している.
\section{今後の課題と展望}
今回の検証で、1人麻雀の和了率において期待和了巡目を扱った方法は和了率の向上の点で有用性を示すことが出来たが、上級者には及んでいない。上級者と本研究の手法の違いには、麻雀における「変化」を考慮するかしないかという点があげられる。上級者はシャンテン数が小さくなる選択肢があったとしても、敢えて長期的な目線でシャンテン数を増やさない選択を行う事がある。本研究においての期待和了巡目はこのような「変化」を考慮していないが、これを考慮することでより高い和了率を出すことができる可能性があると考えられる。また、4人麻雀の成績の結果から、多人数性の影響である鳴きや降りなどの観点を今後は考慮していく必要がある。特に、上級者の4人麻雀の和了のおよそ1/3のは鳴きによるものであるため、鳴きを考慮することで、さらなる和了率の向上が見込めると考えられる。

\nocite{*}