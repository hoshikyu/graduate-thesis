\chapter{先行研究と今回実装する戦略の定義}

\section{麻雀戦略の原則}
従来の麻雀研究によって、大まかなモデルとして、戦略全体に言える原則が存在する。それについて先に書く。

\subsection{局収支論}

局収支論については、点棒状況判断を用いなくても基本的に各局において最大のスコアをやっていけば問題ないという理論である。
実際にそのグラフと理論をみてみよう。

図

このように、点棒状況判断を特段考えなくても、局収支論をベースにすることによって、十分な成績の向上が可能ということが理論的にわかっている。
平均順位と局収支の大きさはほぼ線形に比例すると言っていい。

したがって、今回の検証戦略において点棒状況判断は入れない。

局収支を最大にする戦略として行う。

また麻雀の基本戦略については、大きく分けて攻める戦略と守る戦略が存在する。
次からはそれを分けて述べていく。

\section{攻めの戦略}

まずは攻めの戦略について考察する。

攻める戦略については、従来は様々な要素が考案されていた。麻雀はスコアを高くしていくことが大事となるため、スコアを高くするにはどうしたら良い化が大事となる。
基本的には、打点と速さということになるが、科学する麻雀では打点は二の次でよく、速さのほうが重要であることが大まかに言えている。
実際にその理論をみてみる。

図

したがって、今回は手役については一切考えずに、考えていくことにする。

\subsection{リーチ}
最速で聴牌に向かうという考え方と似ているが、聴牌したら基本的にリーチを打つべきである。
その理由としては〜〜〜であるからである。

参考計算式

\subsection{まとめ}

以上より、麻雀において攻める戦略と言うものを従来の理論的考察を利用して考えると、戦略を簡単にした結果、
「最速で聴牌に向かい、リーチを打つ」 という結論に至る。

したがって、今回攻める戦略について実装する自動うちシステムの攻める戦略については、
「最速で聴牌に向かい、リーチを打つ」ということである。

今までの理論により、これでも十分な成績が見られることが期待できる。

\section{守りの戦略}

\subsection{ベタオリ}

先制リーチに対してはイーシャンテンであっても基本的に降りたほうが有利ということがわかっている。

参考文献:みーにんさんの研究 細かいグラフ

したがって、殆どの場合自分がテンパイしていないのであれば降りたほうが有利である。

\subsection{読みの技術}

読みの技術に関しても、実際にはそこまで大きな影響を与えることがないということが以前の筆者の研究によってわかっている。

(ほしきゅーとcraneの麻雀研究の統計グラフを使って、実際にはそこまで差がないようなことを示す。)

特に、定説は信用できない部分も非常に多く、システムチックに降りたほうがいいと言うことがわかる。
→危険度表

(同じく自分の統計から)

\subsection{まとめ}

以上より、麻雀の守りの戦略を成績影響度の高い要素に簡単に絞ると、戦略としては、
「先制リーチを受けた場合はすべて降りる。その際降りる場合は危険度表に従う」

今までの理論により、これでも十分な成績が見られることが期待できる。