\chapter{攻める戦略について}

麻雀の基本戦略については、大きく分けて攻める戦略と守る戦略が存在する。

まずは攻める戦略について考察する。

攻める戦略については、従来は様々な要素が考案されていた。麻雀はスコアを高くしていくことが大事となるため、スコアを高くするにはどうしたら良い化が大事となる。
基本的には、打点と速さということになるが、科学する麻雀では打点は二の次でよく、速さのほうが重要であることが大まかに言えている。
実際にその理論をみてみる。

図

したがって、今回は手役については一切考えずに、唯一ドラについて(要検討)考えていくことにする。


\subsection{局収支論}

↑麻雀戦略において外観を示すものであるから、攻める戦略と降りる戦略より外側の話で入れたほうがいいかも。

局収支論については、点棒状況判断を用いなくても基本的に各局において最大のスコアをやっていけば問題ないという理論である。
実際にそのグラフと理論をみてみよう。

図

このように、点棒状況判断を特段考えなくても、局収支論をベースにすることによって、十分な成績の向上が可能ということが理論的にわかっている。
平均順位と局収支の大きさはほぼ線形に比例すると言っていい。

したがって、今回の検証戦略において点棒状況判断は入れない。

局収支を最大にする戦略として行ってみる。

\subsection{リーチ}
最速で聴牌に向かうという考え方と似ているが、聴牌したら常にリーチを打つべきである。
その理由としては〜〜〜であるからである。


\subsection{攻める戦略についての結論}

以上より、麻雀において攻める戦略と言うものを従来の理論的考察を利用して考えると、戦略を簡単にした結果、
「最速で聴牌に向かい、リーチを打つ」 という結論に至る。

したがって、今回攻める戦略について実装する自動うちシステムの攻める戦略については、
「最速で聴牌に向かい、リーチを打つ」ということである。

今までの理論により、これでも十分な成績が見られることが期待できる。