\chapter{結論}
\label{chap:conclusion}
\section{本研究のまとめ}
本研究では、各牌姿における期待和了平均順目の評価をモンテカルロ木探索に利用することに
よって、モンテカルロ木探索を麻雀に適用することを提案した。評価として、1人麻雀の和了率
を比較した。先行研究では UCB 1や LinUCB を1人麻雀に適用する例が報告されていたが、こ
れらは和了までのシミュレーションを行っているものであり、探索空間が多く精度が落ちるため
平均プレイヤには及ばなかった。本研究の提案手法では、和了率が平均プレイヤ達することが確
認できた。また、期待平均和了順目をどれほど深くの変化まで計算するかに関しては、深く計算
することでより精度が保たれることが確認できた。ただし、5巡目以降に関しては大きな違いが
見られないことがわかった。4人麻雀の評価については、シャンテン数を下げるように打った先
行研究よりもレートが高いということを示すことが出来た。

\section{本研究の結論}
麻雀において従来の探索方法では探索空間が大きくなりすぎて精度が落ちる問題に対して、期
待平均和了順目を用いて和了への確率を簡潔に評価することにより、探索空間を小さくする方法
が有用であることがわかった。また、期待平均和了順目はおよそ5巡目までの変化を考慮するこ
とで十分な精度が期待できることがわかった。最後に、4人麻雀 AI のアルゴリズムとして、モン
テカルロ木探索を期待平均和了巡目を指標に行う手法についてはある程度性能を期待できること
がわかった。
% 本稿では, 牌譜の局面からの教師あり学習や異なる探索 の結果の共有ができる LinUCB を 1 人麻雀に適用すること を提案した. LinUCB と, 比較手法として UCB1 と事前学 習を利用する手法を 1 人麻雀に適用し, あがった局数を比 較することで評価を行った. その結果いずれの手法も人間 のプレイヤには及ばず, 3 つの手法では UCB1 が最も良い あがり率を出す結果となった. しかし, UCB1 と事前学習 のみを用いる手法の性能が近かったことから, 事前学習が 不完全情報ゲームに対してある程度有効であるということ が分かった. また, Plain LinUCB, 事前学習+LinUCB の 結果から特徴量の設計が不適切であったことが考えられる.
% 今後は特徴量の見直して実験を行い, LinUCB の性能評 価を再度行いたい. 本実験では, LinUCB と事前学習を利用 する手法を分けて評価を行ったが, LinUCB で成果が出る ようになれば, これらを組み合わせることで更に性能が向 上することが期待されるのでこれについても評価を行いた い. また, UCB をモンテカルロ木探索に適用した UCT [8] が成果をあげていることから LinUCB をモンテカルロ木探 索へ適用することも検討している.
\section{今後の課題と展望}
今回の検証では、既存のモンテカルロ木探索を行った研究よりも、1人麻雀の和了率は高くなることを示すことが出来た。しかし、牌譜から教師信号を学習した方法により和了率を上回ることは出来ていない。この原因としては、期待平均和了順目を用いる方法では、ツモを無限思考回数行える場合の平均順目を最小にするように考えていることである。実際の麻雀では、ツモ回数が限られていて、相手の和了によってもその終局までの巡目が異なる。今後の課題としては、期待平均和了順目に対して、残りツモ回数の変数を導入したモデルを構築し、同じようにモンテカルロ木探索を行うことである。このようにすることで、より4人麻雀に近い状態での和了率最大化を目指すことができる。