\chapter{結論}
\label{chap:conclusion}
\section{本研究のまとめ}

本稿では, 牌譜の局面からの教師あり学習や異なる探索 の結果の共有ができる LinUCB を 1 人麻雀に適用すること を提案した. LinUCB と, 比較手法として UCB1 と事前学 習を利用する手法を 1 人麻雀に適用し, あがった局数を比 較することで評価を行った. その結果いずれの手法も人間 のプレイヤには及ばず, 3 つの手法では UCB1 が最も良い あがり率を出す結果となった. しかし, UCB1 と事前学習 のみを用いる手法の性能が近かったことから, 事前学習が 不完全情報ゲームに対してある程度有効であるということ が分かった. また, Plain LinUCB, 事前学習+LinUCB の 結果から特徴量の設計が不適切であったことが考えられる.
今後は特徴量の見直して実験を行い, LinUCB の性能評 価を再度行いたい. 本実験では, LinUCB と事前学習を利用 する手法を分けて評価を行ったが, LinUCB で成果が出る ようになれば, これらを組み合わせることで更に性能が向 上することが期待されるのでこれについても評価を行いた い. また, UCB をモンテカルロ木探索に適用した UCT [8] が成果をあげていることから LinUCB をモンテカルロ木探 索へ適用することも検討している.
参考文献
\section{今後の課題と展望}