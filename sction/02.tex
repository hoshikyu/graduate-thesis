\chapter{本テンプレートの使い方}
\label{chap:howto}

本章では、本テンプレートの具体的な使用方法を解説する。基本的には、{\tt main.tex} を上から順に修正していけばよいだけ。 


\section{テンプレートの構成}

このテンプレートは、表\ref{tb:files}のファイルで構成されている。

\begin{table}[htbp]
  \caption{構成ファイル}
  \label{tb:files}
  \begin{center}\begin{tabular}{c|l}
    \hline
    ファイル名&用途\\\hline\hline
    {\tt main.tex}&メインのファイル。これを編集していく\\\hline
    {\tt thesis.sty}&論文のスタイルを定義したファイル。基本的には手は加えない\\\hline
    {\tt *.tex}&{\tt main.tex}に{\tt include}されるファイル群\\\hline
    {\tt *.eps}&画像ファイル\\\hline
    {\tt main.bib}&参考文献用のBibTeXファイル\\\hline
    {\tt Makefile}&Makefile。次節以降で説明\\\hline
    {\tt .gitignore}&Git用設定ファイル\\\hline
  \end{tabular}\end{center}
\end{table}

\section{コンパイル}
このテンプレートの\LaTeX ファイルをコンパイルしてPDFファイルを生成するには、ターミナルを開いて以下のようにする。

\begin{itembox}[l]{コマンド実行例}
\begin{verbatim}
% make
\end{verbatim}
\end{itembox}

こうすることで、\verb|platex|コマンド、\verb|pbibtex|コマンド、\verb|platex|コマンド2回、\verb|dvipdfmx|コマンドが全て実行され、{\tt main.pdf}が生成される。

コンパイルによって生成されたファイルを全て消すには、以下のようにする。

\begin{itembox}[l]{コマンド実行例}
\begin{verbatim}
% make clean
\end{verbatim}
\end{itembox}

\section{設定}

以下、{\tt main.tex}に対して行うべき設定を、このファイルの中に書いてある順に沿って説明する。

\subsection{論文全体の言語の設定}
\label{sec:lang}

\begin{itembox}[l]{{\tt main.tex}}
\begin{verbatim}
\japanesetrue	% 論文全体を日本語で書く(英語で書くならコメントアウト)
\end{verbatim}
\end{itembox}

ここでは論文全体の言語を設定する。日本語に設定すれば、『章』『目次』『謝辞』などが日本語で出力されて、行頭のインデントなども日本語の仕様になる。英語にした場合は、これらはそれぞれ『Chapter』『Table of Contents』『Acknowledgment』な体裁になる。インデントも行間も、英語用の設定が適用される。

\verb|\japanesetrue| をコメントアウトしなければ日本語に、コメントアウトすれば英語に設定される。


\subsection{余白の設定}

\begin{itembox}[l]{{\tt main.tex}}
\begin{verbatim}
\bindermode	% バインダ用余白設定
\end{verbatim}
\end{itembox}

このテンプレートの出力はA4用紙。ここではこれの四辺の余白を設定する。

最終的にバインダーで綴じて提出する場合、余白を左右対称にしてしまうと、見かけ上のバランスがとても悪くなる。これを解消するため、あらかじめ左側の余白を大きく取っておく。

\verb|\bindermode| をコメントアウトしなければ左綴じ用の余白に、コメントアウトすれば左右対称の余白に設定される。

両面印刷の場合、偶数ページと奇数ページで余白を広くとるべき側が違うので、\verb|documentclass| でこれを設定する。

\begin{itembox}[l]{{\tt main.tex}}
\begin{verbatim}
% 両面印刷の場合。余白を綴じ側に作って右起こし。
\documentclass[a4j,twoside,openright,11pt]{jreport}
% 片面印刷の場合。
%\documentclass[a4j,11pt]{jreport}
\end{verbatim}
\end{itembox}

両面印刷の場合は \verb|twoside| を使用する。\verb|openright| を使うと章のはじまりが必ず右側のページに来るようになる。

\subsection{論文情報の設定}
\label{sec:meta}

\begin{itembox}[l]{{\tt main.tex}}
\begin{verbatim}
% 日本語情報(必要なら)
\jclass  {修士論文}                             % 論文種別
\jtitle    {修士論文用 \LaTeX\ テンプレート}    % タイトル。改行する場合は\\を入れる
\juniv    {慶應義塾大学大学院}                  % 大学名
\jfaculty  {政策・メディア研究科}               % 学部、学科
\jauthor  {ほげ山 ふう助}                       % 著者
\jhyear  {24}                                   % 平成○年度
\jsyear  {2012}                                 % 西暦○年度
\jkeyword  {\LaTeX、テンプレート、修士論文}     % 論文のキーワード
\jproject{インタラクションデザインプロジェクト} %プロジェクト名
\jdate{2013年1月}

% 英語情報(必要なら)
\eclass  {Master's Thesis}                            % 論文種別
\etitle    {A \LaTeX Template for Master Thesis}      % タイトル。改行する場合は\\を入れる
\euniv  {Keio University}                             % 大学名
\efaculty  {Graduate School of Media and Governance}  % 学部、学科
\eauthor  {Fusuke Hogeyama}                           % 著者
\eyear  {2012}                                        % 西暦○年度
\ekeyword  {\LaTeX, Templete, Master Thesis}          % 論文のキーワード
\eproject{Interaction Design Project}                 %プロジェクト名
\edate{January 2013}
\end{verbatim}
\end{itembox}

ここでは論文のタイトルや著者の氏名などのメタデータを記述する。ここで書いたデータは、表紙とアブストラクトのページに使われる。必ずしも日本語と英語の両方を設定しなければいけないわけではなくて、自分が必要とする方だけ記述すればよい。

タイトルが長過ぎる場合は、表紙やアブストラクトのページでは自動で折り返して出力される。もし改行位置を自分で指定したい場合は、その場所に \verb|\\| を入力する。


\section{出力}

\verb|\begin{document}| から \verb|\end{document}| に記述した部分が、実際に{\tt DVI}(最終的には{\tt PDF})ファイルとして出力される。

\subsection{外部ファイルの読み込み({\tt include})}

出力部分の具体的な説明の前に、外部ファイルを読み込む方法を説明する。

\verb|\begin{document}| から \verb|\end{document}| の間では、\verb|\include| コマンドを使うことで、別の {\tt *.tex} ファイルを読み込ませられる。 

\begin{itembox}[l]{{\tt include}しない場合}
\begin{itembox}[l]{{\tt main.tex}}
\begin{verbatim}
\begin{document}
  \begin{jabstract}
  ほげほげ
  \end{jabstract}
\end{document}
\end{verbatim}
\end{itembox}
\end{itembox}

\begin{itembox}[l]{{\tt include}する場合}
\begin{minipage}{0.5\hsize}
\begin{itembox}[l]{{\tt main.tex}}
\begin{verbatim}
\begin{document}
\chapter{序論}
\label{chap:introduction}

論文は序論のようなもので始める。タイトルは序論でも序言でもはじめにでもいいけど、『序論』で始めたら『結論』で終わり、『序言』で始めたら『結言』で終わるようにする。『はじめに』なら『おわりに』で終わる。『序論』で始まって『おわりに』でおわるとか、そういうちぐはぐなのはだめ。

ここでは序論として書く。序論では、研究の背景やら目的やらを書くのが普通。今はテンプレートの説明なので、大して書くことは無い。


\section{背景}

ここではこのテンプレートのオリジナルの作者である @kurokobo の書いたもの\cite{kurokobo10}を引用したい。

\begin{quotation}
ぼくは別に\LaTeX に明るいわけではなくて、この研究室に所属してから初めて触った程度。四年生になってぼく自身が卒業論文を書くことになって、先生は\LaTeX を推奨していたんだけど、テンプレートありますかって聞いたら特にないから作ってほしいとのことだったので、じゃあ作りますよ、という流れ。ぼく自身が使いやすいように、自分が使いながらいろいろ改良をして、こうして公開している。

作成にあたっては、先輩方の卒業論文や主にぐーぐる先生を活用したインターネット上の情報を参考にした。

ただ、卒業論文の体裁は、それぞれの研究室の文化や、担当の指導教員のこだわりも強く影響することも事実。このテンプレートは、『ぼくが所属していた研究室』という、ごくごく限定的でローカルな仕様に沿ったフォーマット――より正確に言えば『ぼくが所属していた研究室ではNGではなかった』フォーマット――というだけのもの。そのあたり、承知の上で使ってほしい。

他の研究室で使う場合は、指導教員の許可を仰ぐほうが確実。
\end{quotation}

筆者はこの卒業論文用のテンプレートを大学院ガイドに例示されている体裁\cite{mag_guide12}に沿うように改造した。これは論文の形式で言えばもっと後ろに書いてあるべきことなのかもしれない。

\section{本文書の構成}

第1章の最後は、文書全体の構成を大まかに書くとよいらしい。

第\ref{chap:introduction}章では本テンプレートの概要みたいなものを書いた。第\ref{chap:howto}章では、本テンプレートの使い方を説明する。第\ref{chap:latex}章で図表や数式の挿入など代表的な\LaTeX コマンドを解説する。第\ref{chap:conclusion}章では、『序論』で始めたら『結論』で終われと書いた手前書かざるを得ないので、なにか結論らしいことを書く。付録として、テンプレートのサンプルになるように無理矢理ゴミを添付する。
 % 01.texをinclude
\end{document}
\end{verbatim}
\end{itembox}
\end{minipage}
\begin{minipage}{0.5\hsize}
\begin{itembox}[l]{{\tt 01.tex}}
\begin{verbatim}
\begin{jabstract}
ほげほげ
\end{jabstract}
\end{verbatim}
\end{itembox}
\end{minipage}
\end{itembox}

{\tt include}しない場合とする場合を比較するとこのとおり。どちらも出力結果は一緒。{\tt include}する場合は、読み込ませたい箇所に、読み込ませたい{\tt *.tex}ファイルの名前を、拡張子を除いて \verb|\include| コマンドで書けばよい。

\verb|\include| コマンドを用いるか用いないかは、たぶん文書量や個人の好みに依る。例えば章ごとに別のファイルにしておけば、修正箇所を探すときの手間が多少は省けるかもしれない。Gitで人と共有しつつ校正を頼むときにもファイルが分かれていたほうがコンフリクトを起こしにくい。


\subsection{表紙の出力}

\begin{itembox}[l]{{\tt main.tex}}
\begin{verbatim}
\ifjapanese
  \jmaketitle    % 表紙(日本語)
\else
  \emaketitle    % 表紙(英語)
\fi
\end{verbatim}
\end{itembox}

最初に、表紙を出力する。

\verb|\jmaketitle| が実行されると日本語の表紙が、\verb|\emaketitle| が実行されると英語の表紙がそれぞれ出力される。日本語の表紙には、第\ref{sec:meta}節で設定したうちの日本語の情報が、英語の表紙には同節で設定したうち英語の情報が、それぞれ参照されて、表記される。

デフォルトでは第\ref{sec:lang}説で設定した言語の表紙のみが出力されるようになっている。

\subsection{アブストラクトの出力}

\begin{itembox}[l]{{\tt main.tex}}
\begin{verbatim}
% ■ アブストラクトの出力 ■
%	◆書式:
%		begin{jabstract}〜end{jabstract}	:日本語のアブストラクト
%		begin{eabstract}〜end{eabstract}	:英語のアブストラクト
%		※ 不要ならばコマンドごと消せば出力されない。

% 日本語のアブストラクト
\begin{jabstract}

ゲームAIによる昨今の成果として、囲碁や将棋などのゲームでは、人間のチャンピオンレベルの実力に至っている。一方、麻雀においては未だそれらに及んでいない。麻雀は多人数不完全情報ゲームに分類され、多人数性による複雑性と不完全情報による不確定性を理由に、適切なモデルを作成することが難しいからである。多人数性の問題では、目的の異なる相手プレイヤーの行動を予測することが難しい。また、不完全情報の問題では、未知の情報を仮定することが必要となるため、期待値による行動の決定を行わなければならない。実現可能なパターン全てを探索して比較することも、探索空間が膨大になりすぎるため非現実的である。

本研究では、以上に述べたような麻雀の性質から、問題を和了に限定し、和了率を高めるアルゴリズムに注目する。和了率は麻雀において得点する頻度を表す指標であり、これを最大化することは成績を上げるために重要である。和了率を高めるための打牌アルゴリズムとして、本研究では期待和了巡目の評価を利用する手法を提案する。期待和了巡目とは、与えられた牌姿において和了までにかかる平均消費巡目を理論値的に概算して求めたものである。これが最も小さくなるような打牌を選択することで、和了率の最大化を目指す。先行研究では単に有効牌の数だけで和了のしやすさを比較していたため、有効牌同士の優劣を精密に比較することができない問題があった。本研究の期待和了巡目を用いた手法では、有効牌をさらにブロックと呼ばれる種類ごとに分類することで、この問題を解決し、和了率を上げることを目的とする。

この手法の効果を確かめる実験として、多人数性を取り除いた1人麻雀の成績と、通常の4人麻雀の成績を、シャンテン数や有効牌を指標としたアルゴリズムや人間プレイヤーと比較した。本手法で構築したアルゴリズムは1人麻雀において、シャンテン数だけで比較するアルゴリズムよりも約10%、有効牌の数だけで比較するアルゴリズムよりも約3%高い和了率を示すことがわかった。4人麻雀では、それぞれよりわずかに和了率とレーティングが上回ったが、統計的に優位な差ではなかった。


\end{jabstract}


% 英語のアブストラクト
\begin{eabstract}

Recent achievements by game AI have led to human champion's ability in games such as Go and Shogi. On the other hand, Mahjong has not yet reached it. Mahjong is categorized as a multi-player incomplete information game, and it is difficult to create an appropriate model for reasons of complexity due to multiplicity and uncertainty due to incomplete information.

In this thesis, we propose a method to use the evaluation of expected winning cycle as batting tile algorithm to raise the rate of winning.

As an experiment to confirm the effectiveness of this method, we compare the performance of one mahjong who removed the multiplayer and the result of ordinary four person mahjong with the algorithm of related research and human player.

It was found that the algorithm constructed by this method shows higher winning rate than the algorithm to compare by merely the number of effective tiles in one person mahjong. In the 4-person mahjong, the rate of winning and ratings were slightly higher, but it was not statistically significant difference.

\end{eabstract}
	% アブストラクト。要独自コマンド、include先参照のこと
\end{verbatim}
\end{itembox}

表紙の次は、アブストラクト。

アブストラクトを出力するには、出力したい位置に、指定のコマンドを用いて文章を書き下せばよい。{\tt main.tex}に直接書いてもよいし、先述した \verb|\include| コマンドを利用して{\tt include}してもよい。

\verb|\begin{jabstract}| から \verb|\end{jabstract}| の間に書いた文章が日本語のアブストラクトとして、\verb|\begin{eabstract}| から \verb|\end{eabstract}| の間に書いた文章が英語のアブストラクトとして、それぞれ独立したページに出力される。

アブストラクトのページには、論文のタイトルやキーワードなどが、第\ref{sec:meta}節で設定した情報をもとにして自動で表記される。

日本語か英語のどちらか一方のみでよい場合は、不要な言語の方のコマンドを削除すればよい。これは、\verb|\begin| と \verb|\end| というコマンド自身も含めて削除する、ということで、\verb|\begin| と \verb|\end| の間を空っぽにするという意味ではないので注意。



\subsection{目次類の出力}
\label{sec:toc}

\begin{itembox}[l]{{\tt main.tex}}
\begin{verbatim}
\tableofcontents	% 目次
\listoffigures		% 表目次
\listoftables		% 図目次
\end{verbatim}
\end{itembox}

アブストラクトの次に、目次。文書の目次、図の目次、表の目次の三種類。

目次類を出力するには、出力したい位置に指定のコマンドを書けばよい。

これらのコマンドは、コンパイル時点での一時ファイル\footnote{{\tt *.toc}、{\tt *.lof}、{\tt *.lot}}の情報を、目次として体裁を整えて出力するもの。一時ファイルは、\verb|\begin{document}| から \verb|\end{document}| の間の章や節、図や表をコンパイルするときに、ついでに情報を取得しておいて生成される。

つまり気をつけなければいけないのは、コンパイルを一回しただけでは、一時ファイルが最新の状態に更新されるだけで、肝心の目次は正しい情報では出力されないということ。目次類を正しい情報で出力するには、最低二回のコンパイルが必要。一回目のコンパイルで一時ファイルが最新の情報に更新されて、二回目のコンパイルで初めて、その最新の一時ファイルの情報をもとに目次が出力される。

だから、文書に何らかの修正をして保存したあとは、最低でも二回、連続してコンパイルしないといけないことに注意する。

図や表を一つも使用していない場合は、目次名のみが書かれた空白のページが出力される。もしこれが不要な場合は、該当するコマンドをコメントアウトすればよい。


\subsection{本文の出力}

\begin{itembox}[l]{{\tt main.tex}}
\begin{verbatim}
\chapter{序論}
\label{chap:introduction}

論文は序論のようなもので始める。タイトルは序論でも序言でもはじめにでもいいけど、『序論』で始めたら『結論』で終わり、『序言』で始めたら『結言』で終わるようにする。『はじめに』なら『おわりに』で終わる。『序論』で始まって『おわりに』でおわるとか、そういうちぐはぐなのはだめ。

ここでは序論として書く。序論では、研究の背景やら目的やらを書くのが普通。今はテンプレートの説明なので、大して書くことは無い。


\section{背景}

ここではこのテンプレートのオリジナルの作者である @kurokobo の書いたもの\cite{kurokobo10}を引用したい。

\begin{quotation}
ぼくは別に\LaTeX に明るいわけではなくて、この研究室に所属してから初めて触った程度。四年生になってぼく自身が卒業論文を書くことになって、先生は\LaTeX を推奨していたんだけど、テンプレートありますかって聞いたら特にないから作ってほしいとのことだったので、じゃあ作りますよ、という流れ。ぼく自身が使いやすいように、自分が使いながらいろいろ改良をして、こうして公開している。

作成にあたっては、先輩方の卒業論文や主にぐーぐる先生を活用したインターネット上の情報を参考にした。

ただ、卒業論文の体裁は、それぞれの研究室の文化や、担当の指導教員のこだわりも強く影響することも事実。このテンプレートは、『ぼくが所属していた研究室』という、ごくごく限定的でローカルな仕様に沿ったフォーマット――より正確に言えば『ぼくが所属していた研究室ではNGではなかった』フォーマット――というだけのもの。そのあたり、承知の上で使ってほしい。

他の研究室で使う場合は、指導教員の許可を仰ぐほうが確実。
\end{quotation}

筆者はこの卒業論文用のテンプレートを大学院ガイドに例示されている体裁\cite{mag_guide12}に沿うように改造した。これは論文の形式で言えばもっと後ろに書いてあるべきことなのかもしれない。

\section{本文書の構成}

第1章の最後は、文書全体の構成を大まかに書くとよいらしい。

第\ref{chap:introduction}章では本テンプレートの概要みたいなものを書いた。第\ref{chap:howto}章では、本テンプレートの使い方を説明する。第\ref{chap:latex}章で図表や数式の挿入など代表的な\LaTeX コマンドを解説する。第\ref{chap:conclusion}章では、『序論』で始めたら『結論』で終われと書いた手前書かざるを得ないので、なにか結論らしいことを書く。付録として、テンプレートのサンプルになるように無理矢理ゴミを添付する。
	% 本文1
\chapter{本テンプレートの使い方}
\label{chap:howto}

本章では、本テンプレートの具体的な使用方法を解説する。基本的には、{\tt main.tex} を上から順に修正していけばよいだけ。 


\section{テンプレートの構成}

このテンプレートは、表\ref{tb:files}のファイルで構成されている。

\begin{table}[htbp]
  \caption{構成ファイル}
  \label{tb:files}
  \begin{center}\begin{tabular}{c|l}
    \hline
    ファイル名&用途\\\hline\hline
    {\tt main.tex}&メインのファイル。これを編集していく\\\hline
    {\tt thesis.sty}&論文のスタイルを定義したファイル。基本的には手は加えない\\\hline
    {\tt *.tex}&{\tt main.tex}に{\tt include}されるファイル群\\\hline
    {\tt *.eps}&画像ファイル\\\hline
    {\tt main.bib}&参考文献用のBibTeXファイル\\\hline
    {\tt Makefile}&Makefile。次節以降で説明\\\hline
    {\tt .gitignore}&Git用設定ファイル\\\hline
  \end{tabular}\end{center}
\end{table}

\section{コンパイル}
このテンプレートの\LaTeX ファイルをコンパイルしてPDFファイルを生成するには、ターミナルを開いて以下のようにする。

\begin{itembox}[l]{コマンド実行例}
\begin{verbatim}
% make
\end{verbatim}
\end{itembox}

こうすることで、\verb|platex|コマンド、\verb|pbibtex|コマンド、\verb|platex|コマンド2回、\verb|dvipdfmx|コマンドが全て実行され、{\tt main.pdf}が生成される。

コンパイルによって生成されたファイルを全て消すには、以下のようにする。

\begin{itembox}[l]{コマンド実行例}
\begin{verbatim}
% make clean
\end{verbatim}
\end{itembox}

\section{設定}

以下、{\tt main.tex}に対して行うべき設定を、このファイルの中に書いてある順に沿って説明する。

\subsection{論文全体の言語の設定}
\label{sec:lang}

\begin{itembox}[l]{{\tt main.tex}}
\begin{verbatim}
\japanesetrue	% 論文全体を日本語で書く(英語で書くならコメントアウト)
\end{verbatim}
\end{itembox}

ここでは論文全体の言語を設定する。日本語に設定すれば、『章』『目次』『謝辞』などが日本語で出力されて、行頭のインデントなども日本語の仕様になる。英語にした場合は、これらはそれぞれ『Chapter』『Table of Contents』『Acknowledgment』な体裁になる。インデントも行間も、英語用の設定が適用される。

\verb|\japanesetrue| をコメントアウトしなければ日本語に、コメントアウトすれば英語に設定される。


\subsection{余白の設定}

\begin{itembox}[l]{{\tt main.tex}}
\begin{verbatim}
\bindermode	% バインダ用余白設定
\end{verbatim}
\end{itembox}

このテンプレートの出力はA4用紙。ここではこれの四辺の余白を設定する。

最終的にバインダーで綴じて提出する場合、余白を左右対称にしてしまうと、見かけ上のバランスがとても悪くなる。これを解消するため、あらかじめ左側の余白を大きく取っておく。

\verb|\bindermode| をコメントアウトしなければ左綴じ用の余白に、コメントアウトすれば左右対称の余白に設定される。

両面印刷の場合、偶数ページと奇数ページで余白を広くとるべき側が違うので、\verb|documentclass| でこれを設定する。

\begin{itembox}[l]{{\tt main.tex}}
\begin{verbatim}
% 両面印刷の場合。余白を綴じ側に作って右起こし。
\documentclass[a4j,twoside,openright,11pt]{jreport}
% 片面印刷の場合。
%\documentclass[a4j,11pt]{jreport}
\end{verbatim}
\end{itembox}

両面印刷の場合は \verb|twoside| を使用する。\verb|openright| を使うと章のはじまりが必ず右側のページに来るようになる。

\subsection{論文情報の設定}
\label{sec:meta}

\begin{itembox}[l]{{\tt main.tex}}
\begin{verbatim}
% 日本語情報(必要なら)
\jclass  {修士論文}                             % 論文種別
\jtitle    {修士論文用 \LaTeX\ テンプレート}    % タイトル。改行する場合は\\を入れる
\juniv    {慶應義塾大学大学院}                  % 大学名
\jfaculty  {政策・メディア研究科}               % 学部、学科
\jauthor  {ほげ山 ふう助}                       % 著者
\jhyear  {24}                                   % 平成○年度
\jsyear  {2012}                                 % 西暦○年度
\jkeyword  {\LaTeX、テンプレート、修士論文}     % 論文のキーワード
\jproject{インタラクションデザインプロジェクト} %プロジェクト名
\jdate{2013年1月}

% 英語情報(必要なら)
\eclass  {Master's Thesis}                            % 論文種別
\etitle    {A \LaTeX Template for Master Thesis}      % タイトル。改行する場合は\\を入れる
\euniv  {Keio University}                             % 大学名
\efaculty  {Graduate School of Media and Governance}  % 学部、学科
\eauthor  {Fusuke Hogeyama}                           % 著者
\eyear  {2012}                                        % 西暦○年度
\ekeyword  {\LaTeX, Templete, Master Thesis}          % 論文のキーワード
\eproject{Interaction Design Project}                 %プロジェクト名
\edate{January 2013}
\end{verbatim}
\end{itembox}

ここでは論文のタイトルや著者の氏名などのメタデータを記述する。ここで書いたデータは、表紙とアブストラクトのページに使われる。必ずしも日本語と英語の両方を設定しなければいけないわけではなくて、自分が必要とする方だけ記述すればよい。

タイトルが長過ぎる場合は、表紙やアブストラクトのページでは自動で折り返して出力される。もし改行位置を自分で指定したい場合は、その場所に \verb|\\| を入力する。


\section{出力}

\verb|\begin{document}| から \verb|\end{document}| に記述した部分が、実際に{\tt DVI}(最終的には{\tt PDF})ファイルとして出力される。

\subsection{外部ファイルの読み込み({\tt include})}

出力部分の具体的な説明の前に、外部ファイルを読み込む方法を説明する。

\verb|\begin{document}| から \verb|\end{document}| の間では、\verb|\include| コマンドを使うことで、別の {\tt *.tex} ファイルを読み込ませられる。 

\begin{itembox}[l]{{\tt include}しない場合}
\begin{itembox}[l]{{\tt main.tex}}
\begin{verbatim}
\begin{document}
  \begin{jabstract}
  ほげほげ
  \end{jabstract}
\end{document}
\end{verbatim}
\end{itembox}
\end{itembox}

\begin{itembox}[l]{{\tt include}する場合}
\begin{minipage}{0.5\hsize}
\begin{itembox}[l]{{\tt main.tex}}
\begin{verbatim}
\begin{document}
\chapter{序論}
\label{chap:introduction}

論文は序論のようなもので始める。タイトルは序論でも序言でもはじめにでもいいけど、『序論』で始めたら『結論』で終わり、『序言』で始めたら『結言』で終わるようにする。『はじめに』なら『おわりに』で終わる。『序論』で始まって『おわりに』でおわるとか、そういうちぐはぐなのはだめ。

ここでは序論として書く。序論では、研究の背景やら目的やらを書くのが普通。今はテンプレートの説明なので、大して書くことは無い。


\section{背景}

ここではこのテンプレートのオリジナルの作者である @kurokobo の書いたもの\cite{kurokobo10}を引用したい。

\begin{quotation}
ぼくは別に\LaTeX に明るいわけではなくて、この研究室に所属してから初めて触った程度。四年生になってぼく自身が卒業論文を書くことになって、先生は\LaTeX を推奨していたんだけど、テンプレートありますかって聞いたら特にないから作ってほしいとのことだったので、じゃあ作りますよ、という流れ。ぼく自身が使いやすいように、自分が使いながらいろいろ改良をして、こうして公開している。

作成にあたっては、先輩方の卒業論文や主にぐーぐる先生を活用したインターネット上の情報を参考にした。

ただ、卒業論文の体裁は、それぞれの研究室の文化や、担当の指導教員のこだわりも強く影響することも事実。このテンプレートは、『ぼくが所属していた研究室』という、ごくごく限定的でローカルな仕様に沿ったフォーマット――より正確に言えば『ぼくが所属していた研究室ではNGではなかった』フォーマット――というだけのもの。そのあたり、承知の上で使ってほしい。

他の研究室で使う場合は、指導教員の許可を仰ぐほうが確実。
\end{quotation}

筆者はこの卒業論文用のテンプレートを大学院ガイドに例示されている体裁\cite{mag_guide12}に沿うように改造した。これは論文の形式で言えばもっと後ろに書いてあるべきことなのかもしれない。

\section{本文書の構成}

第1章の最後は、文書全体の構成を大まかに書くとよいらしい。

第\ref{chap:introduction}章では本テンプレートの概要みたいなものを書いた。第\ref{chap:howto}章では、本テンプレートの使い方を説明する。第\ref{chap:latex}章で図表や数式の挿入など代表的な\LaTeX コマンドを解説する。第\ref{chap:conclusion}章では、『序論』で始めたら『結論』で終われと書いた手前書かざるを得ないので、なにか結論らしいことを書く。付録として、テンプレートのサンプルになるように無理矢理ゴミを添付する。
 % 01.texをinclude
\end{document}
\end{verbatim}
\end{itembox}
\end{minipage}
\begin{minipage}{0.5\hsize}
\begin{itembox}[l]{{\tt 01.tex}}
\begin{verbatim}
\begin{jabstract}
ほげほげ
\end{jabstract}
\end{verbatim}
\end{itembox}
\end{minipage}
\end{itembox}

{\tt include}しない場合とする場合を比較するとこのとおり。どちらも出力結果は一緒。{\tt include}する場合は、読み込ませたい箇所に、読み込ませたい{\tt *.tex}ファイルの名前を、拡張子を除いて \verb|\include| コマンドで書けばよい。

\verb|\include| コマンドを用いるか用いないかは、たぶん文書量や個人の好みに依る。例えば章ごとに別のファイルにしておけば、修正箇所を探すときの手間が多少は省けるかもしれない。Gitで人と共有しつつ校正を頼むときにもファイルが分かれていたほうがコンフリクトを起こしにくい。


\subsection{表紙の出力}

\begin{itembox}[l]{{\tt main.tex}}
\begin{verbatim}
\ifjapanese
  \jmaketitle    % 表紙(日本語)
\else
  \emaketitle    % 表紙(英語)
\fi
\end{verbatim}
\end{itembox}

最初に、表紙を出力する。

\verb|\jmaketitle| が実行されると日本語の表紙が、\verb|\emaketitle| が実行されると英語の表紙がそれぞれ出力される。日本語の表紙には、第\ref{sec:meta}節で設定したうちの日本語の情報が、英語の表紙には同節で設定したうち英語の情報が、それぞれ参照されて、表記される。

デフォルトでは第\ref{sec:lang}説で設定した言語の表紙のみが出力されるようになっている。

\subsection{アブストラクトの出力}

\begin{itembox}[l]{{\tt main.tex}}
\begin{verbatim}
% ■ アブストラクトの出力 ■
%	◆書式:
%		begin{jabstract}〜end{jabstract}	:日本語のアブストラクト
%		begin{eabstract}〜end{eabstract}	:英語のアブストラクト
%		※ 不要ならばコマンドごと消せば出力されない。

% 日本語のアブストラクト
\begin{jabstract}

ゲームAIによる昨今の成果として、囲碁や将棋などのゲームでは、人間のチャンピオンレベルの実力に至っている。一方、麻雀においては未だそれらに及んでいない。麻雀は多人数不完全情報ゲームに分類され、多人数性による複雑性と不完全情報による不確定性を理由に、適切なモデルを作成することが難しいからである。多人数性の問題では、目的の異なる相手プレイヤーの行動を予測することが難しい。また、不完全情報の問題では、未知の情報を仮定することが必要となるため、期待値による行動の決定を行わなければならない。実現可能なパターン全てを探索して比較することも、探索空間が膨大になりすぎるため非現実的である。

本研究では、以上に述べたような麻雀の性質から、問題を和了に限定し、和了率を高めるアルゴリズムに注目する。和了率は麻雀において得点する頻度を表す指標であり、これを最大化することは成績を上げるために重要である。和了率を高めるための打牌アルゴリズムとして、本研究では期待和了巡目の評価を利用する手法を提案する。期待和了巡目とは、与えられた牌姿において和了までにかかる平均消費巡目を理論値的に概算して求めたものである。これが最も小さくなるような打牌を選択することで、和了率の最大化を目指す。先行研究では単に有効牌の数だけで和了のしやすさを比較していたため、有効牌同士の優劣を精密に比較することができない問題があった。本研究の期待和了巡目を用いた手法では、有効牌をさらにブロックと呼ばれる種類ごとに分類することで、この問題を解決し、和了率を上げることを目的とする。

この手法の効果を確かめる実験として、多人数性を取り除いた1人麻雀の成績と、通常の4人麻雀の成績を、シャンテン数や有効牌を指標としたアルゴリズムや人間プレイヤーと比較した。本手法で構築したアルゴリズムは1人麻雀において、シャンテン数だけで比較するアルゴリズムよりも約10%、有効牌の数だけで比較するアルゴリズムよりも約3%高い和了率を示すことがわかった。4人麻雀では、それぞれよりわずかに和了率とレーティングが上回ったが、統計的に優位な差ではなかった。


\end{jabstract}


% 英語のアブストラクト
\begin{eabstract}

Recent achievements by game AI have led to human champion's ability in games such as Go and Shogi. On the other hand, Mahjong has not yet reached it. Mahjong is categorized as a multi-player incomplete information game, and it is difficult to create an appropriate model for reasons of complexity due to multiplicity and uncertainty due to incomplete information.

In this thesis, we propose a method to use the evaluation of expected winning cycle as batting tile algorithm to raise the rate of winning.

As an experiment to confirm the effectiveness of this method, we compare the performance of one mahjong who removed the multiplayer and the result of ordinary four person mahjong with the algorithm of related research and human player.

It was found that the algorithm constructed by this method shows higher winning rate than the algorithm to compare by merely the number of effective tiles in one person mahjong. In the 4-person mahjong, the rate of winning and ratings were slightly higher, but it was not statistically significant difference.

\end{eabstract}
	% アブストラクト。要独自コマンド、include先参照のこと
\end{verbatim}
\end{itembox}

表紙の次は、アブストラクト。

アブストラクトを出力するには、出力したい位置に、指定のコマンドを用いて文章を書き下せばよい。{\tt main.tex}に直接書いてもよいし、先述した \verb|\include| コマンドを利用して{\tt include}してもよい。

\verb|\begin{jabstract}| から \verb|\end{jabstract}| の間に書いた文章が日本語のアブストラクトとして、\verb|\begin{eabstract}| から \verb|\end{eabstract}| の間に書いた文章が英語のアブストラクトとして、それぞれ独立したページに出力される。

アブストラクトのページには、論文のタイトルやキーワードなどが、第\ref{sec:meta}節で設定した情報をもとにして自動で表記される。

日本語か英語のどちらか一方のみでよい場合は、不要な言語の方のコマンドを削除すればよい。これは、\verb|\begin| と \verb|\end| というコマンド自身も含めて削除する、ということで、\verb|\begin| と \verb|\end| の間を空っぽにするという意味ではないので注意。



\subsection{目次類の出力}
\label{sec:toc}

\begin{itembox}[l]{{\tt main.tex}}
\begin{verbatim}
\tableofcontents	% 目次
\listoffigures		% 表目次
\listoftables		% 図目次
\end{verbatim}
\end{itembox}

アブストラクトの次に、目次。文書の目次、図の目次、表の目次の三種類。

目次類を出力するには、出力したい位置に指定のコマンドを書けばよい。

これらのコマンドは、コンパイル時点での一時ファイル\footnote{{\tt *.toc}、{\tt *.lof}、{\tt *.lot}}の情報を、目次として体裁を整えて出力するもの。一時ファイルは、\verb|\begin{document}| から \verb|\end{document}| の間の章や節、図や表をコンパイルするときに、ついでに情報を取得しておいて生成される。

つまり気をつけなければいけないのは、コンパイルを一回しただけでは、一時ファイルが最新の状態に更新されるだけで、肝心の目次は正しい情報では出力されないということ。目次類を正しい情報で出力するには、最低二回のコンパイルが必要。一回目のコンパイルで一時ファイルが最新の情報に更新されて、二回目のコンパイルで初めて、その最新の一時ファイルの情報をもとに目次が出力される。

だから、文書に何らかの修正をして保存したあとは、最低でも二回、連続してコンパイルしないといけないことに注意する。

図や表を一つも使用していない場合は、目次名のみが書かれた空白のページが出力される。もしこれが不要な場合は、該当するコマンドをコメントアウトすればよい。


\subsection{本文の出力}

\begin{itembox}[l]{{\tt main.tex}}
\begin{verbatim}
\chapter{序論}
\label{chap:introduction}

論文は序論のようなもので始める。タイトルは序論でも序言でもはじめにでもいいけど、『序論』で始めたら『結論』で終わり、『序言』で始めたら『結言』で終わるようにする。『はじめに』なら『おわりに』で終わる。『序論』で始まって『おわりに』でおわるとか、そういうちぐはぐなのはだめ。

ここでは序論として書く。序論では、研究の背景やら目的やらを書くのが普通。今はテンプレートの説明なので、大して書くことは無い。


\section{背景}

ここではこのテンプレートのオリジナルの作者である @kurokobo の書いたもの\cite{kurokobo10}を引用したい。

\begin{quotation}
ぼくは別に\LaTeX に明るいわけではなくて、この研究室に所属してから初めて触った程度。四年生になってぼく自身が卒業論文を書くことになって、先生は\LaTeX を推奨していたんだけど、テンプレートありますかって聞いたら特にないから作ってほしいとのことだったので、じゃあ作りますよ、という流れ。ぼく自身が使いやすいように、自分が使いながらいろいろ改良をして、こうして公開している。

作成にあたっては、先輩方の卒業論文や主にぐーぐる先生を活用したインターネット上の情報を参考にした。

ただ、卒業論文の体裁は、それぞれの研究室の文化や、担当の指導教員のこだわりも強く影響することも事実。このテンプレートは、『ぼくが所属していた研究室』という、ごくごく限定的でローカルな仕様に沿ったフォーマット――より正確に言えば『ぼくが所属していた研究室ではNGではなかった』フォーマット――というだけのもの。そのあたり、承知の上で使ってほしい。

他の研究室で使う場合は、指導教員の許可を仰ぐほうが確実。
\end{quotation}

筆者はこの卒業論文用のテンプレートを大学院ガイドに例示されている体裁\cite{mag_guide12}に沿うように改造した。これは論文の形式で言えばもっと後ろに書いてあるべきことなのかもしれない。

\section{本文書の構成}

第1章の最後は、文書全体の構成を大まかに書くとよいらしい。

第\ref{chap:introduction}章では本テンプレートの概要みたいなものを書いた。第\ref{chap:howto}章では、本テンプレートの使い方を説明する。第\ref{chap:latex}章で図表や数式の挿入など代表的な\LaTeX コマンドを解説する。第\ref{chap:conclusion}章では、『序論』で始めたら『結論』で終われと書いた手前書かざるを得ないので、なにか結論らしいことを書く。付録として、テンプレートのサンプルになるように無理矢理ゴミを添付する。
	% 本文1
\chapter{本テンプレートの使い方}
\label{chap:howto}

本章では、本テンプレートの具体的な使用方法を解説する。基本的には、{\tt main.tex} を上から順に修正していけばよいだけ。 


\section{テンプレートの構成}

このテンプレートは、表\ref{tb:files}のファイルで構成されている。

\begin{table}[htbp]
  \caption{構成ファイル}
  \label{tb:files}
  \begin{center}\begin{tabular}{c|l}
    \hline
    ファイル名&用途\\\hline\hline
    {\tt main.tex}&メインのファイル。これを編集していく\\\hline
    {\tt thesis.sty}&論文のスタイルを定義したファイル。基本的には手は加えない\\\hline
    {\tt *.tex}&{\tt main.tex}に{\tt include}されるファイル群\\\hline
    {\tt *.eps}&画像ファイル\\\hline
    {\tt main.bib}&参考文献用のBibTeXファイル\\\hline
    {\tt Makefile}&Makefile。次節以降で説明\\\hline
    {\tt .gitignore}&Git用設定ファイル\\\hline
  \end{tabular}\end{center}
\end{table}

\section{コンパイル}
このテンプレートの\LaTeX ファイルをコンパイルしてPDFファイルを生成するには、ターミナルを開いて以下のようにする。

\begin{itembox}[l]{コマンド実行例}
\begin{verbatim}
% make
\end{verbatim}
\end{itembox}

こうすることで、\verb|platex|コマンド、\verb|pbibtex|コマンド、\verb|platex|コマンド2回、\verb|dvipdfmx|コマンドが全て実行され、{\tt main.pdf}が生成される。

コンパイルによって生成されたファイルを全て消すには、以下のようにする。

\begin{itembox}[l]{コマンド実行例}
\begin{verbatim}
% make clean
\end{verbatim}
\end{itembox}

\section{設定}

以下、{\tt main.tex}に対して行うべき設定を、このファイルの中に書いてある順に沿って説明する。

\subsection{論文全体の言語の設定}
\label{sec:lang}

\begin{itembox}[l]{{\tt main.tex}}
\begin{verbatim}
\japanesetrue	% 論文全体を日本語で書く(英語で書くならコメントアウト)
\end{verbatim}
\end{itembox}

ここでは論文全体の言語を設定する。日本語に設定すれば、『章』『目次』『謝辞』などが日本語で出力されて、行頭のインデントなども日本語の仕様になる。英語にした場合は、これらはそれぞれ『Chapter』『Table of Contents』『Acknowledgment』な体裁になる。インデントも行間も、英語用の設定が適用される。

\verb|\japanesetrue| をコメントアウトしなければ日本語に、コメントアウトすれば英語に設定される。


\subsection{余白の設定}

\begin{itembox}[l]{{\tt main.tex}}
\begin{verbatim}
\bindermode	% バインダ用余白設定
\end{verbatim}
\end{itembox}

このテンプレートの出力はA4用紙。ここではこれの四辺の余白を設定する。

最終的にバインダーで綴じて提出する場合、余白を左右対称にしてしまうと、見かけ上のバランスがとても悪くなる。これを解消するため、あらかじめ左側の余白を大きく取っておく。

\verb|\bindermode| をコメントアウトしなければ左綴じ用の余白に、コメントアウトすれば左右対称の余白に設定される。

両面印刷の場合、偶数ページと奇数ページで余白を広くとるべき側が違うので、\verb|documentclass| でこれを設定する。

\begin{itembox}[l]{{\tt main.tex}}
\begin{verbatim}
% 両面印刷の場合。余白を綴じ側に作って右起こし。
\documentclass[a4j,twoside,openright,11pt]{jreport}
% 片面印刷の場合。
%\documentclass[a4j,11pt]{jreport}
\end{verbatim}
\end{itembox}

両面印刷の場合は \verb|twoside| を使用する。\verb|openright| を使うと章のはじまりが必ず右側のページに来るようになる。

\subsection{論文情報の設定}
\label{sec:meta}

\begin{itembox}[l]{{\tt main.tex}}
\begin{verbatim}
% 日本語情報(必要なら)
\jclass  {修士論文}                             % 論文種別
\jtitle    {修士論文用 \LaTeX\ テンプレート}    % タイトル。改行する場合は\\を入れる
\juniv    {慶應義塾大学大学院}                  % 大学名
\jfaculty  {政策・メディア研究科}               % 学部、学科
\jauthor  {ほげ山 ふう助}                       % 著者
\jhyear  {24}                                   % 平成○年度
\jsyear  {2012}                                 % 西暦○年度
\jkeyword  {\LaTeX、テンプレート、修士論文}     % 論文のキーワード
\jproject{インタラクションデザインプロジェクト} %プロジェクト名
\jdate{2013年1月}

% 英語情報(必要なら)
\eclass  {Master's Thesis}                            % 論文種別
\etitle    {A \LaTeX Template for Master Thesis}      % タイトル。改行する場合は\\を入れる
\euniv  {Keio University}                             % 大学名
\efaculty  {Graduate School of Media and Governance}  % 学部、学科
\eauthor  {Fusuke Hogeyama}                           % 著者
\eyear  {2012}                                        % 西暦○年度
\ekeyword  {\LaTeX, Templete, Master Thesis}          % 論文のキーワード
\eproject{Interaction Design Project}                 %プロジェクト名
\edate{January 2013}
\end{verbatim}
\end{itembox}

ここでは論文のタイトルや著者の氏名などのメタデータを記述する。ここで書いたデータは、表紙とアブストラクトのページに使われる。必ずしも日本語と英語の両方を設定しなければいけないわけではなくて、自分が必要とする方だけ記述すればよい。

タイトルが長過ぎる場合は、表紙やアブストラクトのページでは自動で折り返して出力される。もし改行位置を自分で指定したい場合は、その場所に \verb|\\| を入力する。


\section{出力}

\verb|\begin{document}| から \verb|\end{document}| に記述した部分が、実際に{\tt DVI}(最終的には{\tt PDF})ファイルとして出力される。

\subsection{外部ファイルの読み込み({\tt include})}

出力部分の具体的な説明の前に、外部ファイルを読み込む方法を説明する。

\verb|\begin{document}| から \verb|\end{document}| の間では、\verb|\include| コマンドを使うことで、別の {\tt *.tex} ファイルを読み込ませられる。 

\begin{itembox}[l]{{\tt include}しない場合}
\begin{itembox}[l]{{\tt main.tex}}
\begin{verbatim}
\begin{document}
  \begin{jabstract}
  ほげほげ
  \end{jabstract}
\end{document}
\end{verbatim}
\end{itembox}
\end{itembox}

\begin{itembox}[l]{{\tt include}する場合}
\begin{minipage}{0.5\hsize}
\begin{itembox}[l]{{\tt main.tex}}
\begin{verbatim}
\begin{document}
\chapter{序論}
\label{chap:introduction}

論文は序論のようなもので始める。タイトルは序論でも序言でもはじめにでもいいけど、『序論』で始めたら『結論』で終わり、『序言』で始めたら『結言』で終わるようにする。『はじめに』なら『おわりに』で終わる。『序論』で始まって『おわりに』でおわるとか、そういうちぐはぐなのはだめ。

ここでは序論として書く。序論では、研究の背景やら目的やらを書くのが普通。今はテンプレートの説明なので、大して書くことは無い。


\section{背景}

ここではこのテンプレートのオリジナルの作者である @kurokobo の書いたもの\cite{kurokobo10}を引用したい。

\begin{quotation}
ぼくは別に\LaTeX に明るいわけではなくて、この研究室に所属してから初めて触った程度。四年生になってぼく自身が卒業論文を書くことになって、先生は\LaTeX を推奨していたんだけど、テンプレートありますかって聞いたら特にないから作ってほしいとのことだったので、じゃあ作りますよ、という流れ。ぼく自身が使いやすいように、自分が使いながらいろいろ改良をして、こうして公開している。

作成にあたっては、先輩方の卒業論文や主にぐーぐる先生を活用したインターネット上の情報を参考にした。

ただ、卒業論文の体裁は、それぞれの研究室の文化や、担当の指導教員のこだわりも強く影響することも事実。このテンプレートは、『ぼくが所属していた研究室』という、ごくごく限定的でローカルな仕様に沿ったフォーマット――より正確に言えば『ぼくが所属していた研究室ではNGではなかった』フォーマット――というだけのもの。そのあたり、承知の上で使ってほしい。

他の研究室で使う場合は、指導教員の許可を仰ぐほうが確実。
\end{quotation}

筆者はこの卒業論文用のテンプレートを大学院ガイドに例示されている体裁\cite{mag_guide12}に沿うように改造した。これは論文の形式で言えばもっと後ろに書いてあるべきことなのかもしれない。

\section{本文書の構成}

第1章の最後は、文書全体の構成を大まかに書くとよいらしい。

第\ref{chap:introduction}章では本テンプレートの概要みたいなものを書いた。第\ref{chap:howto}章では、本テンプレートの使い方を説明する。第\ref{chap:latex}章で図表や数式の挿入など代表的な\LaTeX コマンドを解説する。第\ref{chap:conclusion}章では、『序論』で始めたら『結論』で終われと書いた手前書かざるを得ないので、なにか結論らしいことを書く。付録として、テンプレートのサンプルになるように無理矢理ゴミを添付する。
 % 01.texをinclude
\end{document}
\end{verbatim}
\end{itembox}
\end{minipage}
\begin{minipage}{0.5\hsize}
\begin{itembox}[l]{{\tt 01.tex}}
\begin{verbatim}
\begin{jabstract}
ほげほげ
\end{jabstract}
\end{verbatim}
\end{itembox}
\end{minipage}
\end{itembox}

{\tt include}しない場合とする場合を比較するとこのとおり。どちらも出力結果は一緒。{\tt include}する場合は、読み込ませたい箇所に、読み込ませたい{\tt *.tex}ファイルの名前を、拡張子を除いて \verb|\include| コマンドで書けばよい。

\verb|\include| コマンドを用いるか用いないかは、たぶん文書量や個人の好みに依る。例えば章ごとに別のファイルにしておけば、修正箇所を探すときの手間が多少は省けるかもしれない。Gitで人と共有しつつ校正を頼むときにもファイルが分かれていたほうがコンフリクトを起こしにくい。


\subsection{表紙の出力}

\begin{itembox}[l]{{\tt main.tex}}
\begin{verbatim}
\ifjapanese
  \jmaketitle    % 表紙(日本語)
\else
  \emaketitle    % 表紙(英語)
\fi
\end{verbatim}
\end{itembox}

最初に、表紙を出力する。

\verb|\jmaketitle| が実行されると日本語の表紙が、\verb|\emaketitle| が実行されると英語の表紙がそれぞれ出力される。日本語の表紙には、第\ref{sec:meta}節で設定したうちの日本語の情報が、英語の表紙には同節で設定したうち英語の情報が、それぞれ参照されて、表記される。

デフォルトでは第\ref{sec:lang}説で設定した言語の表紙のみが出力されるようになっている。

\subsection{アブストラクトの出力}

\begin{itembox}[l]{{\tt main.tex}}
\begin{verbatim}
% ■ アブストラクトの出力 ■
%	◆書式:
%		begin{jabstract}〜end{jabstract}	:日本語のアブストラクト
%		begin{eabstract}〜end{eabstract}	:英語のアブストラクト
%		※ 不要ならばコマンドごと消せば出力されない。

% 日本語のアブストラクト
\begin{jabstract}

ゲームAIによる昨今の成果として、囲碁や将棋などのゲームでは、人間のチャンピオンレベルの実力に至っている。一方、麻雀においては未だそれらに及んでいない。麻雀は多人数不完全情報ゲームに分類され、多人数性による複雑性と不完全情報による不確定性を理由に、適切なモデルを作成することが難しいからである。多人数性の問題では、目的の異なる相手プレイヤーの行動を予測することが難しい。また、不完全情報の問題では、未知の情報を仮定することが必要となるため、期待値による行動の決定を行わなければならない。実現可能なパターン全てを探索して比較することも、探索空間が膨大になりすぎるため非現実的である。

本研究では、以上に述べたような麻雀の性質から、問題を和了に限定し、和了率を高めるアルゴリズムに注目する。和了率は麻雀において得点する頻度を表す指標であり、これを最大化することは成績を上げるために重要である。和了率を高めるための打牌アルゴリズムとして、本研究では期待和了巡目の評価を利用する手法を提案する。期待和了巡目とは、与えられた牌姿において和了までにかかる平均消費巡目を理論値的に概算して求めたものである。これが最も小さくなるような打牌を選択することで、和了率の最大化を目指す。先行研究では単に有効牌の数だけで和了のしやすさを比較していたため、有効牌同士の優劣を精密に比較することができない問題があった。本研究の期待和了巡目を用いた手法では、有効牌をさらにブロックと呼ばれる種類ごとに分類することで、この問題を解決し、和了率を上げることを目的とする。

この手法の効果を確かめる実験として、多人数性を取り除いた1人麻雀の成績と、通常の4人麻雀の成績を、シャンテン数や有効牌を指標としたアルゴリズムや人間プレイヤーと比較した。本手法で構築したアルゴリズムは1人麻雀において、シャンテン数だけで比較するアルゴリズムよりも約10%、有効牌の数だけで比較するアルゴリズムよりも約3%高い和了率を示すことがわかった。4人麻雀では、それぞれよりわずかに和了率とレーティングが上回ったが、統計的に優位な差ではなかった。


\end{jabstract}


% 英語のアブストラクト
\begin{eabstract}

Recent achievements by game AI have led to human champion's ability in games such as Go and Shogi. On the other hand, Mahjong has not yet reached it. Mahjong is categorized as a multi-player incomplete information game, and it is difficult to create an appropriate model for reasons of complexity due to multiplicity and uncertainty due to incomplete information.

In this thesis, we propose a method to use the evaluation of expected winning cycle as batting tile algorithm to raise the rate of winning.

As an experiment to confirm the effectiveness of this method, we compare the performance of one mahjong who removed the multiplayer and the result of ordinary four person mahjong with the algorithm of related research and human player.

It was found that the algorithm constructed by this method shows higher winning rate than the algorithm to compare by merely the number of effective tiles in one person mahjong. In the 4-person mahjong, the rate of winning and ratings were slightly higher, but it was not statistically significant difference.

\end{eabstract}
	% アブストラクト。要独自コマンド、include先参照のこと
\end{verbatim}
\end{itembox}

表紙の次は、アブストラクト。

アブストラクトを出力するには、出力したい位置に、指定のコマンドを用いて文章を書き下せばよい。{\tt main.tex}に直接書いてもよいし、先述した \verb|\include| コマンドを利用して{\tt include}してもよい。

\verb|\begin{jabstract}| から \verb|\end{jabstract}| の間に書いた文章が日本語のアブストラクトとして、\verb|\begin{eabstract}| から \verb|\end{eabstract}| の間に書いた文章が英語のアブストラクトとして、それぞれ独立したページに出力される。

アブストラクトのページには、論文のタイトルやキーワードなどが、第\ref{sec:meta}節で設定した情報をもとにして自動で表記される。

日本語か英語のどちらか一方のみでよい場合は、不要な言語の方のコマンドを削除すればよい。これは、\verb|\begin| と \verb|\end| というコマンド自身も含めて削除する、ということで、\verb|\begin| と \verb|\end| の間を空っぽにするという意味ではないので注意。



\subsection{目次類の出力}
\label{sec:toc}

\begin{itembox}[l]{{\tt main.tex}}
\begin{verbatim}
\tableofcontents	% 目次
\listoffigures		% 表目次
\listoftables		% 図目次
\end{verbatim}
\end{itembox}

アブストラクトの次に、目次。文書の目次、図の目次、表の目次の三種類。

目次類を出力するには、出力したい位置に指定のコマンドを書けばよい。

これらのコマンドは、コンパイル時点での一時ファイル\footnote{{\tt *.toc}、{\tt *.lof}、{\tt *.lot}}の情報を、目次として体裁を整えて出力するもの。一時ファイルは、\verb|\begin{document}| から \verb|\end{document}| の間の章や節、図や表をコンパイルするときに、ついでに情報を取得しておいて生成される。

つまり気をつけなければいけないのは、コンパイルを一回しただけでは、一時ファイルが最新の状態に更新されるだけで、肝心の目次は正しい情報では出力されないということ。目次類を正しい情報で出力するには、最低二回のコンパイルが必要。一回目のコンパイルで一時ファイルが最新の情報に更新されて、二回目のコンパイルで初めて、その最新の一時ファイルの情報をもとに目次が出力される。

だから、文書に何らかの修正をして保存したあとは、最低でも二回、連続してコンパイルしないといけないことに注意する。

図や表を一つも使用していない場合は、目次名のみが書かれた空白のページが出力される。もしこれが不要な場合は、該当するコマンドをコメントアウトすればよい。


\subsection{本文の出力}

\begin{itembox}[l]{{\tt main.tex}}
\begin{verbatim}
\chapter{序論}
\label{chap:introduction}

論文は序論のようなもので始める。タイトルは序論でも序言でもはじめにでもいいけど、『序論』で始めたら『結論』で終わり、『序言』で始めたら『結言』で終わるようにする。『はじめに』なら『おわりに』で終わる。『序論』で始まって『おわりに』でおわるとか、そういうちぐはぐなのはだめ。

ここでは序論として書く。序論では、研究の背景やら目的やらを書くのが普通。今はテンプレートの説明なので、大して書くことは無い。


\section{背景}

ここではこのテンプレートのオリジナルの作者である @kurokobo の書いたもの\cite{kurokobo10}を引用したい。

\begin{quotation}
ぼくは別に\LaTeX に明るいわけではなくて、この研究室に所属してから初めて触った程度。四年生になってぼく自身が卒業論文を書くことになって、先生は\LaTeX を推奨していたんだけど、テンプレートありますかって聞いたら特にないから作ってほしいとのことだったので、じゃあ作りますよ、という流れ。ぼく自身が使いやすいように、自分が使いながらいろいろ改良をして、こうして公開している。

作成にあたっては、先輩方の卒業論文や主にぐーぐる先生を活用したインターネット上の情報を参考にした。

ただ、卒業論文の体裁は、それぞれの研究室の文化や、担当の指導教員のこだわりも強く影響することも事実。このテンプレートは、『ぼくが所属していた研究室』という、ごくごく限定的でローカルな仕様に沿ったフォーマット――より正確に言えば『ぼくが所属していた研究室ではNGではなかった』フォーマット――というだけのもの。そのあたり、承知の上で使ってほしい。

他の研究室で使う場合は、指導教員の許可を仰ぐほうが確実。
\end{quotation}

筆者はこの卒業論文用のテンプレートを大学院ガイドに例示されている体裁\cite{mag_guide12}に沿うように改造した。これは論文の形式で言えばもっと後ろに書いてあるべきことなのかもしれない。

\section{本文書の構成}

第1章の最後は、文書全体の構成を大まかに書くとよいらしい。

第\ref{chap:introduction}章では本テンプレートの概要みたいなものを書いた。第\ref{chap:howto}章では、本テンプレートの使い方を説明する。第\ref{chap:latex}章で図表や数式の挿入など代表的な\LaTeX コマンドを解説する。第\ref{chap:conclusion}章では、『序論』で始めたら『結論』で終われと書いた手前書かざるを得ないので、なにか結論らしいことを書く。付録として、テンプレートのサンプルになるように無理矢理ゴミを添付する。
	% 本文1
\chapter{本テンプレートの使い方}
\label{chap:howto}

本章では、本テンプレートの具体的な使用方法を解説する。基本的には、{\tt main.tex} を上から順に修正していけばよいだけ。 


\section{テンプレートの構成}

このテンプレートは、表\ref{tb:files}のファイルで構成されている。

\begin{table}[htbp]
  \caption{構成ファイル}
  \label{tb:files}
  \begin{center}\begin{tabular}{c|l}
    \hline
    ファイル名&用途\\\hline\hline
    {\tt main.tex}&メインのファイル。これを編集していく\\\hline
    {\tt thesis.sty}&論文のスタイルを定義したファイル。基本的には手は加えない\\\hline
    {\tt *.tex}&{\tt main.tex}に{\tt include}されるファイル群\\\hline
    {\tt *.eps}&画像ファイル\\\hline
    {\tt main.bib}&参考文献用のBibTeXファイル\\\hline
    {\tt Makefile}&Makefile。次節以降で説明\\\hline
    {\tt .gitignore}&Git用設定ファイル\\\hline
  \end{tabular}\end{center}
\end{table}

\section{コンパイル}
このテンプレートの\LaTeX ファイルをコンパイルしてPDFファイルを生成するには、ターミナルを開いて以下のようにする。

\begin{itembox}[l]{コマンド実行例}
\begin{verbatim}
% make
\end{verbatim}
\end{itembox}

こうすることで、\verb|platex|コマンド、\verb|pbibtex|コマンド、\verb|platex|コマンド2回、\verb|dvipdfmx|コマンドが全て実行され、{\tt main.pdf}が生成される。

コンパイルによって生成されたファイルを全て消すには、以下のようにする。

\begin{itembox}[l]{コマンド実行例}
\begin{verbatim}
% make clean
\end{verbatim}
\end{itembox}

\section{設定}

以下、{\tt main.tex}に対して行うべき設定を、このファイルの中に書いてある順に沿って説明する。

\subsection{論文全体の言語の設定}
\label{sec:lang}

\begin{itembox}[l]{{\tt main.tex}}
\begin{verbatim}
\japanesetrue	% 論文全体を日本語で書く(英語で書くならコメントアウト)
\end{verbatim}
\end{itembox}

ここでは論文全体の言語を設定する。日本語に設定すれば、『章』『目次』『謝辞』などが日本語で出力されて、行頭のインデントなども日本語の仕様になる。英語にした場合は、これらはそれぞれ『Chapter』『Table of Contents』『Acknowledgment』な体裁になる。インデントも行間も、英語用の設定が適用される。

\verb|\japanesetrue| をコメントアウトしなければ日本語に、コメントアウトすれば英語に設定される。


\subsection{余白の設定}

\begin{itembox}[l]{{\tt main.tex}}
\begin{verbatim}
\bindermode	% バインダ用余白設定
\end{verbatim}
\end{itembox}

このテンプレートの出力はA4用紙。ここではこれの四辺の余白を設定する。

最終的にバインダーで綴じて提出する場合、余白を左右対称にしてしまうと、見かけ上のバランスがとても悪くなる。これを解消するため、あらかじめ左側の余白を大きく取っておく。

\verb|\bindermode| をコメントアウトしなければ左綴じ用の余白に、コメントアウトすれば左右対称の余白に設定される。

両面印刷の場合、偶数ページと奇数ページで余白を広くとるべき側が違うので、\verb|documentclass| でこれを設定する。

\begin{itembox}[l]{{\tt main.tex}}
\begin{verbatim}
% 両面印刷の場合。余白を綴じ側に作って右起こし。
\documentclass[a4j,twoside,openright,11pt]{jreport}
% 片面印刷の場合。
%\documentclass[a4j,11pt]{jreport}
\end{verbatim}
\end{itembox}

両面印刷の場合は \verb|twoside| を使用する。\verb|openright| を使うと章のはじまりが必ず右側のページに来るようになる。

\subsection{論文情報の設定}
\label{sec:meta}

\begin{itembox}[l]{{\tt main.tex}}
\begin{verbatim}
% 日本語情報(必要なら)
\jclass  {修士論文}                             % 論文種別
\jtitle    {修士論文用 \LaTeX\ テンプレート}    % タイトル。改行する場合は\\を入れる
\juniv    {慶應義塾大学大学院}                  % 大学名
\jfaculty  {政策・メディア研究科}               % 学部、学科
\jauthor  {ほげ山 ふう助}                       % 著者
\jhyear  {24}                                   % 平成○年度
\jsyear  {2012}                                 % 西暦○年度
\jkeyword  {\LaTeX、テンプレート、修士論文}     % 論文のキーワード
\jproject{インタラクションデザインプロジェクト} %プロジェクト名
\jdate{2013年1月}

% 英語情報(必要なら)
\eclass  {Master's Thesis}                            % 論文種別
\etitle    {A \LaTeX Template for Master Thesis}      % タイトル。改行する場合は\\を入れる
\euniv  {Keio University}                             % 大学名
\efaculty  {Graduate School of Media and Governance}  % 学部、学科
\eauthor  {Fusuke Hogeyama}                           % 著者
\eyear  {2012}                                        % 西暦○年度
\ekeyword  {\LaTeX, Templete, Master Thesis}          % 論文のキーワード
\eproject{Interaction Design Project}                 %プロジェクト名
\edate{January 2013}
\end{verbatim}
\end{itembox}

ここでは論文のタイトルや著者の氏名などのメタデータを記述する。ここで書いたデータは、表紙とアブストラクトのページに使われる。必ずしも日本語と英語の両方を設定しなければいけないわけではなくて、自分が必要とする方だけ記述すればよい。

タイトルが長過ぎる場合は、表紙やアブストラクトのページでは自動で折り返して出力される。もし改行位置を自分で指定したい場合は、その場所に \verb|\\| を入力する。


\section{出力}

\verb|\begin{document}| から \verb|\end{document}| に記述した部分が、実際に{\tt DVI}(最終的には{\tt PDF})ファイルとして出力される。

\subsection{外部ファイルの読み込み({\tt include})}

出力部分の具体的な説明の前に、外部ファイルを読み込む方法を説明する。

\verb|\begin{document}| から \verb|\end{document}| の間では、\verb|\include| コマンドを使うことで、別の {\tt *.tex} ファイルを読み込ませられる。 

\begin{itembox}[l]{{\tt include}しない場合}
\begin{itembox}[l]{{\tt main.tex}}
\begin{verbatim}
\begin{document}
  \begin{jabstract}
  ほげほげ
  \end{jabstract}
\end{document}
\end{verbatim}
\end{itembox}
\end{itembox}

\begin{itembox}[l]{{\tt include}する場合}
\begin{minipage}{0.5\hsize}
\begin{itembox}[l]{{\tt main.tex}}
\begin{verbatim}
\begin{document}
\include{01} % 01.texをinclude
\end{document}
\end{verbatim}
\end{itembox}
\end{minipage}
\begin{minipage}{0.5\hsize}
\begin{itembox}[l]{{\tt 01.tex}}
\begin{verbatim}
\begin{jabstract}
ほげほげ
\end{jabstract}
\end{verbatim}
\end{itembox}
\end{minipage}
\end{itembox}

{\tt include}しない場合とする場合を比較するとこのとおり。どちらも出力結果は一緒。{\tt include}する場合は、読み込ませたい箇所に、読み込ませたい{\tt *.tex}ファイルの名前を、拡張子を除いて \verb|\include| コマンドで書けばよい。

\verb|\include| コマンドを用いるか用いないかは、たぶん文書量や個人の好みに依る。例えば章ごとに別のファイルにしておけば、修正箇所を探すときの手間が多少は省けるかもしれない。Gitで人と共有しつつ校正を頼むときにもファイルが分かれていたほうがコンフリクトを起こしにくい。


\subsection{表紙の出力}

\begin{itembox}[l]{{\tt main.tex}}
\begin{verbatim}
\ifjapanese
  \jmaketitle    % 表紙(日本語)
\else
  \emaketitle    % 表紙(英語)
\fi
\end{verbatim}
\end{itembox}

最初に、表紙を出力する。

\verb|\jmaketitle| が実行されると日本語の表紙が、\verb|\emaketitle| が実行されると英語の表紙がそれぞれ出力される。日本語の表紙には、第\ref{sec:meta}節で設定したうちの日本語の情報が、英語の表紙には同節で設定したうち英語の情報が、それぞれ参照されて、表記される。

デフォルトでは第\ref{sec:lang}説で設定した言語の表紙のみが出力されるようになっている。

\subsection{アブストラクトの出力}

\begin{itembox}[l]{{\tt main.tex}}
\begin{verbatim}
\include{00_abstract}	% アブストラクト。要独自コマンド、include先参照のこと
\end{verbatim}
\end{itembox}

表紙の次は、アブストラクト。

アブストラクトを出力するには、出力したい位置に、指定のコマンドを用いて文章を書き下せばよい。{\tt main.tex}に直接書いてもよいし、先述した \verb|\include| コマンドを利用して{\tt include}してもよい。

\verb|\begin{jabstract}| から \verb|\end{jabstract}| の間に書いた文章が日本語のアブストラクトとして、\verb|\begin{eabstract}| から \verb|\end{eabstract}| の間に書いた文章が英語のアブストラクトとして、それぞれ独立したページに出力される。

アブストラクトのページには、論文のタイトルやキーワードなどが、第\ref{sec:meta}節で設定した情報をもとにして自動で表記される。

日本語か英語のどちらか一方のみでよい場合は、不要な言語の方のコマンドを削除すればよい。これは、\verb|\begin| と \verb|\end| というコマンド自身も含めて削除する、ということで、\verb|\begin| と \verb|\end| の間を空っぽにするという意味ではないので注意。



\subsection{目次類の出力}
\label{sec:toc}

\begin{itembox}[l]{{\tt main.tex}}
\begin{verbatim}
\tableofcontents	% 目次
\listoffigures		% 表目次
\listoftables		% 図目次
\end{verbatim}
\end{itembox}

アブストラクトの次に、目次。文書の目次、図の目次、表の目次の三種類。

目次類を出力するには、出力したい位置に指定のコマンドを書けばよい。

これらのコマンドは、コンパイル時点での一時ファイル\footnote{{\tt *.toc}、{\tt *.lof}、{\tt *.lot}}の情報を、目次として体裁を整えて出力するもの。一時ファイルは、\verb|\begin{document}| から \verb|\end{document}| の間の章や節、図や表をコンパイルするときに、ついでに情報を取得しておいて生成される。

つまり気をつけなければいけないのは、コンパイルを一回しただけでは、一時ファイルが最新の状態に更新されるだけで、肝心の目次は正しい情報では出力されないということ。目次類を正しい情報で出力するには、最低二回のコンパイルが必要。一回目のコンパイルで一時ファイルが最新の情報に更新されて、二回目のコンパイルで初めて、その最新の一時ファイルの情報をもとに目次が出力される。

だから、文書に何らかの修正をして保存したあとは、最低でも二回、連続してコンパイルしないといけないことに注意する。

図や表を一つも使用していない場合は、目次名のみが書かれた空白のページが出力される。もしこれが不要な場合は、該当するコマンドをコメントアウトすればよい。


\subsection{本文の出力}

\begin{itembox}[l]{{\tt main.tex}}
\begin{verbatim}
\include{01}	% 本文1
\include{02}	% 本文2
\include{03}	% 本文3
\include{04}	% 本文4
\end{verbatim}
\end{itembox}

目次に続いて、論文のメイン、本文を記述する。アブストラクトと同様で、{\tt main.tex}に直接書くか、\verb|\include| コマンドを利用して別に用意したファイルを{\tt include}する。

本文の書き方は、第\ref{chap:latex}章で詳しく説明する。


\subsection{謝辞の出力}

\begin{itembox}[l]{{\tt main.tex}}
\begin{verbatim}
\include{90_acknowledgment}	% 謝辞。要独自コマンド、include先参照のこと
\end{verbatim}
\end{itembox}

本文のあとには、謝辞を出力する。\verb|begin{acknowledgment}| から \verb|end{acknowledgment}| の間に書いた文章が、謝辞として独立したページに出力される。アブストラクトや本文と同じで、{\tt main.tex}に直接書いてもよいし、\verb|\include| コマンドを利用して{\tt include}してもよい。


\subsection{参考文献の出力}

\begin{itembox}[l]{{\tt main.tex}}
\begin{verbatim}
\include{91_bibliography}	% 参考文献。要独自コマンド、include先参照のこと
\end{verbatim}
\end{itembox}

謝辞に続いて、参考文献を出力する。

参考文献リストは、\verb|\begin{bib}| から \verb|\end{bib}| の間に、\verb|\bibitem| コマンドを使って書く。

BibTeXを使う場合は、以下のようにする。

\begin{itembox}[l]{{\tt 91\_bibliography.tex}}
\begin{verbatim}
\begin{bib}[100]
\bibliography{main}
\end{bib}
\end{verbatim}
\end{itembox}

こうすると、\verb|main.bib|から使用した参考文献のみを抽出して出力してくれる。\verb|main.bib|の中身は以下のようになっていて、気の利いた論文検索サイトであればBibTeXをコピペできるようになっているので簡単に作れるはず。


\begin{itembox}[l]{{\tt 91\_bibliography.tex}}
\begin{verbatim}
@article{hoge09,
    author  = "ほげ山太郎 and ほげ山次郎",
    yomi    = "ほげやまたろう",
    title   = "ほげほげ理論のHCI分野への応用",
    journal = "ほげほげ学会論文誌",
    volume  = "31",
    number  = "3",
    pages   = "194-201",
    year    = "2009",
}
@inproceedings{hoge08,
    author     = "Taro Hogeyama and Jiro Hogeyama",
    title      = "The Theory of Hoge",
    booktitle  = "The Proceedings of The Hoge Society",
    year       = "2008"
}
\end{verbatim}
\end{itembox}


以下は、BibTeXを使わないで手で書く例。

\begin{itembox}[l]{{\tt 91\_bibliography.tex}}
\begin{verbatim}
@article{hoge09,
    author  = "ほげ山太郎 and ほげ山次郎",
    yomi    = "ほげやまたろう",
    title   = "ほげほげ理論のHCI分野への応用",
    journal = "ほげほげ学会論文誌",
    volume  = "31",
    number  = "3",
    pages   = "194-201",
    year    = "2009",
}
@inproceedings{hoge08,
    author     = "Taro Hogeyama and Jiro Hogeyama",
    title      = "The Theory of Hoge",
    booktitle  = "The Proceedings of The Hoge Society",
    year       = "2008"
}
\end{verbatim}
\end{itembox}


英語の文献の場合、慣例的に書誌名をイタリック体にすることが多いらしい。

\begin{itembox}[l]{{\tt 91\_bibliography.tex}}
\begin{verbatim}
\begin{bib}[100]
\begin{thebibliography}{#1}
% \bibitem{参照用名称}
%   著者名: 
%   \newblock 文献名,
%   \newblock 書誌情報,出版年.

\bibitem{hoge09}
  ほげ山太郎,ほげ山次郎:
  \newblock ほげほげ理論のHCI分野への応用,
  \newblock ほげほげ学会論文誌,Vol.31,No.3,pp.194-201,2009.

\bibitem{hoge08}
  Taro Hogeyama, Jiro Hogeyama:
  \newblock The Theory of Hoge,
  \newblock {\it The Proceedings of The Hoge Society}, 2008.
\end{thebibliography}
\end{bib}
\end{verbatim}
\end{itembox}

\verb|\bibitem| コマンド中、参照用名称は、本文から参考文献を参照するときに使うので、忘れずに書いておく。参照文献を本文中に参照するときには、\verb|\cite{参照用名称}| のように書けばよい。例えば、この文の末尾には \verb|\cite{hoge09}| と書いてあるので、自動で対応する番号が振られる\cite{hoge09}\cite{hoge08}。

参考文献リストの番号付けと、本文で参照したときの番号の挿入は、全部が自動で行われる。ただしこれも、第\ref{sec:toc}節で説明した目次の出力と同じで、一時ファイルを生成してからの挿入なので、正しく出力するには最低でも二回のコンパイルが必要。BibTeXを使用する場合は、\verb|platex|コマンドのあと\verb|pbibtex|コマンドを実行し、さらに2回\verb|platex|コマンドを実行するといいらしい。



\subsection{付録の出力}

\begin{itembox}[l]{{\tt main.tex}}
\begin{verbatim}
\appendix
\include{92_appendix}		% 付録
\end{verbatim}
\end{itembox}

必要であれば、論文の最後には付録を出力する。

\verb|\appendix| コマンド以降に書いたものは、すべて付録として扱われる。付録部分の書き方は通常の本文とまったく同じで、\verb|\appendix| コマンド以降に書くだけで勝手に付録用の体裁で出力される。
	% 本文2
\chapter{自動打ちシステムの設計・実装}
\label{chap:implementation}
ここはほとんど今までのtermやORFポスターで書いたようなことを書きます。
実装についてどのようなことをしたのか、システム、コードレベルでの話。
\section{システム構成}
\section{パケット受信}
\section{情報整理}
\section{打牌アルゴリズム実装}
\subsection{シャンテン数アルゴリズム}	% 本文3
\chapter{評価}
\label{chap:conclusion}

評価方法としては、実際に天鳳一般卓で以上に述べた戦略を実装した自動うちシステムを実際に対戦させ、成績を取る。
その際取るパラメータとして、「平均順位」「和了率」「放銃率」などのパラメーターが基本的である。


\section{自作アルゴリズム同士の比較}


オリの技術に関しては非常に重要であるが、実際のところオリの技術というのは成績にどう影響するのかということを、自信のベースラインとの比較によって検証したいと思う

オリを実装していない全ツmy自動打ちシステム   VS  オリを実装したmy自動打ちシステムの比較 


また、統計的検証により、信頼区間やウェルチ検定などで、どれくらいの差があるかということが証明できる。
ただし、オリを実装すれば実際のところ成績が改善することは当たり前なので、本当に自分の降りだけでいいのか?というのを比較するのは次の評価のセクションで考える。

\section{関連研究との比較}

評価方法としては、爆打が同じように天鳳一般卓で打った成績が存在するので、それに対して比較してみることがいいかもしれない。
実際のところ、爆打はあらゆる要素を比較して機械学習してモデルケースを作っているが、実際それがどの程度の影響を及ぼしているのか?を比較したい。

爆打にも、 ベースライン と 提案手法 が存在するので、それをそれぞれ比較することになるかも。これについてはもう少し考えないといけない。

仮説では、自分の簡単な戦略でも十分に平均的なプレイヤとしての成績が期待できるのでそれが実際どの程度の差があるか、各要素を比較することはどれだけ重要なのか、を数値的に明らかにしたいと考えている。



	% 本文4
\end{verbatim}
\end{itembox}

目次に続いて、論文のメイン、本文を記述する。アブストラクトと同様で、{\tt main.tex}に直接書くか、\verb|\include| コマンドを利用して別に用意したファイルを{\tt include}する。

本文の書き方は、第\ref{chap:latex}章で詳しく説明する。


\subsection{謝辞の出力}

\begin{itembox}[l]{{\tt main.tex}}
\begin{verbatim}
\begin{acknowledgment}

このテンプレートを改造するにあたって、@kurokoboとインターネット上のいくつかの修士論文などを参考にしました。感謝いたします。

\end{acknowledgment}
	% 謝辞。要独自コマンド、include先参照のこと
\end{verbatim}
\end{itembox}

本文のあとには、謝辞を出力する。\verb|begin{acknowledgment}| から \verb|end{acknowledgment}| の間に書いた文章が、謝辞として独立したページに出力される。アブストラクトや本文と同じで、{\tt main.tex}に直接書いてもよいし、\verb|\include| コマンドを利用して{\tt include}してもよい。


\subsection{参考文献の出力}

\begin{itembox}[l]{{\tt main.tex}}
\begin{verbatim}

\begin{bib}[100]
% BibTeXを使う場合
\bibliography{bib/main}

%\begin{thebibliography}{#1}
%
%  \bibitem{参照用名称}
%    著者名:
%    \newblock 文献名,
%    \newblock 書誌情報,出版年.
%
% \bibitem{hoge09}
%   ほげ山太郎,ほげ山次郎:
%   \newblock ほげほげ理論のHCI分野への応用,
%   \newblock ほげほげ学会論文誌,Vol.31,No.3,pp.194-201,2009.
%
% \bibitem{hoge08}
%   Taro Hogeyama, Jiro Hogeyama:
%   \newblock The Theory of Hoge,
%   \newblock {\it The Proceedings of The Hoge Society}, 2008.
%
%\end{thebibliography}

\end{bib}
	% 参考文献。要独自コマンド、include先参照のこと
\end{verbatim}
\end{itembox}

謝辞に続いて、参考文献を出力する。

参考文献リストは、\verb|\begin{bib}| から \verb|\end{bib}| の間に、\verb|\bibitem| コマンドを使って書く。

BibTeXを使う場合は、以下のようにする。

\begin{itembox}[l]{{\tt 91\_bibliography.tex}}
\begin{verbatim}
\begin{bib}[100]
\bibliography{main}
\end{bib}
\end{verbatim}
\end{itembox}

こうすると、\verb|main.bib|から使用した参考文献のみを抽出して出力してくれる。\verb|main.bib|の中身は以下のようになっていて、気の利いた論文検索サイトであればBibTeXをコピペできるようになっているので簡単に作れるはず。


\begin{itembox}[l]{{\tt 91\_bibliography.tex}}
\begin{verbatim}
@article{hoge09,
    author  = "ほげ山太郎 and ほげ山次郎",
    yomi    = "ほげやまたろう",
    title   = "ほげほげ理論のHCI分野への応用",
    journal = "ほげほげ学会論文誌",
    volume  = "31",
    number  = "3",
    pages   = "194-201",
    year    = "2009",
}
@inproceedings{hoge08,
    author     = "Taro Hogeyama and Jiro Hogeyama",
    title      = "The Theory of Hoge",
    booktitle  = "The Proceedings of The Hoge Society",
    year       = "2008"
}
\end{verbatim}
\end{itembox}


以下は、BibTeXを使わないで手で書く例。

\begin{itembox}[l]{{\tt 91\_bibliography.tex}}
\begin{verbatim}
@article{hoge09,
    author  = "ほげ山太郎 and ほげ山次郎",
    yomi    = "ほげやまたろう",
    title   = "ほげほげ理論のHCI分野への応用",
    journal = "ほげほげ学会論文誌",
    volume  = "31",
    number  = "3",
    pages   = "194-201",
    year    = "2009",
}
@inproceedings{hoge08,
    author     = "Taro Hogeyama and Jiro Hogeyama",
    title      = "The Theory of Hoge",
    booktitle  = "The Proceedings of The Hoge Society",
    year       = "2008"
}
\end{verbatim}
\end{itembox}


英語の文献の場合、慣例的に書誌名をイタリック体にすることが多いらしい。

\begin{itembox}[l]{{\tt 91\_bibliography.tex}}
\begin{verbatim}
\begin{bib}[100]
\begin{thebibliography}{#1}
% \bibitem{参照用名称}
%   著者名: 
%   \newblock 文献名,
%   \newblock 書誌情報,出版年.

\bibitem{hoge09}
  ほげ山太郎,ほげ山次郎:
  \newblock ほげほげ理論のHCI分野への応用,
  \newblock ほげほげ学会論文誌,Vol.31,No.3,pp.194-201,2009.

\bibitem{hoge08}
  Taro Hogeyama, Jiro Hogeyama:
  \newblock The Theory of Hoge,
  \newblock {\it The Proceedings of The Hoge Society}, 2008.
\end{thebibliography}
\end{bib}
\end{verbatim}
\end{itembox}

\verb|\bibitem| コマンド中、参照用名称は、本文から参考文献を参照するときに使うので、忘れずに書いておく。参照文献を本文中に参照するときには、\verb|\cite{参照用名称}| のように書けばよい。例えば、この文の末尾には \verb|\cite{hoge09}| と書いてあるので、自動で対応する番号が振られる\cite{hoge09}\cite{hoge08}。

参考文献リストの番号付けと、本文で参照したときの番号の挿入は、全部が自動で行われる。ただしこれも、第\ref{sec:toc}節で説明した目次の出力と同じで、一時ファイルを生成してからの挿入なので、正しく出力するには最低でも二回のコンパイルが必要。BibTeXを使用する場合は、\verb|platex|コマンドのあと\verb|pbibtex|コマンドを実行し、さらに2回\verb|platex|コマンドを実行するといいらしい。



\subsection{付録の出力}

\begin{itembox}[l]{{\tt main.tex}}
\begin{verbatim}
\appendix
\chapter{付録の例}

付録を無理矢理出力させるため、てきとうなことを書く。

\section{ほげ}

コマンドは本文と一緒。

\subsection{ふー}

本文と一緒。

\section{ほげほげ}

本文と一緒。

\subsection{ふーふー}

本文と一緒。
		% 付録
\end{verbatim}
\end{itembox}

必要であれば、論文の最後には付録を出力する。

\verb|\appendix| コマンド以降に書いたものは、すべて付録として扱われる。付録部分の書き方は通常の本文とまったく同じで、\verb|\appendix| コマンド以降に書くだけで勝手に付録用の体裁で出力される。
	% 本文2
\chapter{自動打ちシステムの設計・実装}
\label{chap:implementation}
ここはほとんど今までのtermやORFポスターで書いたようなことを書きます。
実装についてどのようなことをしたのか、システム、コードレベルでの話。
\section{システム構成}
\section{パケット受信}
\section{情報整理}
\section{打牌アルゴリズム実装}
\subsection{シャンテン数アルゴリズム}	% 本文3
\chapter{評価}
\label{chap:conclusion}

評価方法としては、実際に天鳳一般卓で以上に述べた戦略を実装した自動うちシステムを実際に対戦させ、成績を取る。
その際取るパラメータとして、「平均順位」「和了率」「放銃率」などのパラメーターが基本的である。


\section{自作アルゴリズム同士の比較}


オリの技術に関しては非常に重要であるが、実際のところオリの技術というのは成績にどう影響するのかということを、自信のベースラインとの比較によって検証したいと思う

オリを実装していない全ツmy自動打ちシステム   VS  オリを実装したmy自動打ちシステムの比較 


また、統計的検証により、信頼区間やウェルチ検定などで、どれくらいの差があるかということが証明できる。
ただし、オリを実装すれば実際のところ成績が改善することは当たり前なので、本当に自分の降りだけでいいのか?というのを比較するのは次の評価のセクションで考える。

\section{関連研究との比較}

評価方法としては、爆打が同じように天鳳一般卓で打った成績が存在するので、それに対して比較してみることがいいかもしれない。
実際のところ、爆打はあらゆる要素を比較して機械学習してモデルケースを作っているが、実際それがどの程度の影響を及ぼしているのか?を比較したい。

爆打にも、 ベースライン と 提案手法 が存在するので、それをそれぞれ比較することになるかも。これについてはもう少し考えないといけない。

仮説では、自分の簡単な戦略でも十分に平均的なプレイヤとしての成績が期待できるのでそれが実際どの程度の差があるか、各要素を比較することはどれだけ重要なのか、を数値的に明らかにしたいと考えている。



	% 本文4
\end{verbatim}
\end{itembox}

目次に続いて、論文のメイン、本文を記述する。アブストラクトと同様で、{\tt main.tex}に直接書くか、\verb|\include| コマンドを利用して別に用意したファイルを{\tt include}する。

本文の書き方は、第\ref{chap:latex}章で詳しく説明する。


\subsection{謝辞の出力}

\begin{itembox}[l]{{\tt main.tex}}
\begin{verbatim}
\begin{acknowledgment}

このテンプレートを改造するにあたって、@kurokoboとインターネット上のいくつかの修士論文などを参考にしました。感謝いたします。

\end{acknowledgment}
	% 謝辞。要独自コマンド、include先参照のこと
\end{verbatim}
\end{itembox}

本文のあとには、謝辞を出力する。\verb|begin{acknowledgment}| から \verb|end{acknowledgment}| の間に書いた文章が、謝辞として独立したページに出力される。アブストラクトや本文と同じで、{\tt main.tex}に直接書いてもよいし、\verb|\include| コマンドを利用して{\tt include}してもよい。


\subsection{参考文献の出力}

\begin{itembox}[l]{{\tt main.tex}}
\begin{verbatim}

\begin{bib}[100]
% BibTeXを使う場合
\bibliography{bib/main}

%\begin{thebibliography}{#1}
%
%  \bibitem{参照用名称}
%    著者名:
%    \newblock 文献名,
%    \newblock 書誌情報,出版年.
%
% \bibitem{hoge09}
%   ほげ山太郎,ほげ山次郎:
%   \newblock ほげほげ理論のHCI分野への応用,
%   \newblock ほげほげ学会論文誌,Vol.31,No.3,pp.194-201,2009.
%
% \bibitem{hoge08}
%   Taro Hogeyama, Jiro Hogeyama:
%   \newblock The Theory of Hoge,
%   \newblock {\it The Proceedings of The Hoge Society}, 2008.
%
%\end{thebibliography}

\end{bib}
	% 参考文献。要独自コマンド、include先参照のこと
\end{verbatim}
\end{itembox}

謝辞に続いて、参考文献を出力する。

参考文献リストは、\verb|\begin{bib}| から \verb|\end{bib}| の間に、\verb|\bibitem| コマンドを使って書く。

BibTeXを使う場合は、以下のようにする。

\begin{itembox}[l]{{\tt 91\_bibliography.tex}}
\begin{verbatim}
\begin{bib}[100]
\bibliography{main}
\end{bib}
\end{verbatim}
\end{itembox}

こうすると、\verb|main.bib|から使用した参考文献のみを抽出して出力してくれる。\verb|main.bib|の中身は以下のようになっていて、気の利いた論文検索サイトであればBibTeXをコピペできるようになっているので簡単に作れるはず。


\begin{itembox}[l]{{\tt 91\_bibliography.tex}}
\begin{verbatim}
@article{hoge09,
    author  = "ほげ山太郎 and ほげ山次郎",
    yomi    = "ほげやまたろう",
    title   = "ほげほげ理論のHCI分野への応用",
    journal = "ほげほげ学会論文誌",
    volume  = "31",
    number  = "3",
    pages   = "194-201",
    year    = "2009",
}
@inproceedings{hoge08,
    author     = "Taro Hogeyama and Jiro Hogeyama",
    title      = "The Theory of Hoge",
    booktitle  = "The Proceedings of The Hoge Society",
    year       = "2008"
}
\end{verbatim}
\end{itembox}


以下は、BibTeXを使わないで手で書く例。

\begin{itembox}[l]{{\tt 91\_bibliography.tex}}
\begin{verbatim}
@article{hoge09,
    author  = "ほげ山太郎 and ほげ山次郎",
    yomi    = "ほげやまたろう",
    title   = "ほげほげ理論のHCI分野への応用",
    journal = "ほげほげ学会論文誌",
    volume  = "31",
    number  = "3",
    pages   = "194-201",
    year    = "2009",
}
@inproceedings{hoge08,
    author     = "Taro Hogeyama and Jiro Hogeyama",
    title      = "The Theory of Hoge",
    booktitle  = "The Proceedings of The Hoge Society",
    year       = "2008"
}
\end{verbatim}
\end{itembox}


英語の文献の場合、慣例的に書誌名をイタリック体にすることが多いらしい。

\begin{itembox}[l]{{\tt 91\_bibliography.tex}}
\begin{verbatim}
\begin{bib}[100]
\begin{thebibliography}{#1}
% \bibitem{参照用名称}
%   著者名: 
%   \newblock 文献名,
%   \newblock 書誌情報,出版年.

\bibitem{hoge09}
  ほげ山太郎,ほげ山次郎:
  \newblock ほげほげ理論のHCI分野への応用,
  \newblock ほげほげ学会論文誌,Vol.31,No.3,pp.194-201,2009.

\bibitem{hoge08}
  Taro Hogeyama, Jiro Hogeyama:
  \newblock The Theory of Hoge,
  \newblock {\it The Proceedings of The Hoge Society}, 2008.
\end{thebibliography}
\end{bib}
\end{verbatim}
\end{itembox}

\verb|\bibitem| コマンド中、参照用名称は、本文から参考文献を参照するときに使うので、忘れずに書いておく。参照文献を本文中に参照するときには、\verb|\cite{参照用名称}| のように書けばよい。例えば、この文の末尾には \verb|\cite{hoge09}| と書いてあるので、自動で対応する番号が振られる\cite{hoge09}\cite{hoge08}。

参考文献リストの番号付けと、本文で参照したときの番号の挿入は、全部が自動で行われる。ただしこれも、第\ref{sec:toc}節で説明した目次の出力と同じで、一時ファイルを生成してからの挿入なので、正しく出力するには最低でも二回のコンパイルが必要。BibTeXを使用する場合は、\verb|platex|コマンドのあと\verb|pbibtex|コマンドを実行し、さらに2回\verb|platex|コマンドを実行するといいらしい。



\subsection{付録の出力}

\begin{itembox}[l]{{\tt main.tex}}
\begin{verbatim}
\appendix
\chapter{付録の例}

付録を無理矢理出力させるため、てきとうなことを書く。

\section{ほげ}

コマンドは本文と一緒。

\subsection{ふー}

本文と一緒。

\section{ほげほげ}

本文と一緒。

\subsection{ふーふー}

本文と一緒。
		% 付録
\end{verbatim}
\end{itembox}

必要であれば、論文の最後には付録を出力する。

\verb|\appendix| コマンド以降に書いたものは、すべて付録として扱われる。付録部分の書き方は通常の本文とまったく同じで、\verb|\appendix| コマンド以降に書くだけで勝手に付録用の体裁で出力される。
	% 本文2
\chapter{自動打ちシステムの設計・実装}
\label{chap:implementation}
ここはほとんど今までのtermやORFポスターで書いたようなことを書きます。
実装についてどのようなことをしたのか、システム、コードレベルでの話。
\section{システム構成}
\section{パケット受信}
\section{情報整理}
\section{打牌アルゴリズム実装}
\subsection{シャンテン数アルゴリズム}	% 本文3
\chapter{評価}
\label{chap:conclusion}

評価方法としては、実際に天鳳一般卓で以上に述べた戦略を実装した自動うちシステムを実際に対戦させ、成績を取る。
その際取るパラメータとして、「平均順位」「和了率」「放銃率」などのパラメーターが基本的である。


\section{自作アルゴリズム同士の比較}


オリの技術に関しては非常に重要であるが、実際のところオリの技術というのは成績にどう影響するのかということを、自信のベースラインとの比較によって検証したいと思う

オリを実装していない全ツmy自動打ちシステム   VS  オリを実装したmy自動打ちシステムの比較 


また、統計的検証により、信頼区間やウェルチ検定などで、どれくらいの差があるかということが証明できる。
ただし、オリを実装すれば実際のところ成績が改善することは当たり前なので、本当に自分の降りだけでいいのか?というのを比較するのは次の評価のセクションで考える。

\section{関連研究との比較}

評価方法としては、爆打が同じように天鳳一般卓で打った成績が存在するので、それに対して比較してみることがいいかもしれない。
実際のところ、爆打はあらゆる要素を比較して機械学習してモデルケースを作っているが、実際それがどの程度の影響を及ぼしているのか?を比較したい。

爆打にも、 ベースライン と 提案手法 が存在するので、それをそれぞれ比較することになるかも。これについてはもう少し考えないといけない。

仮説では、自分の簡単な戦略でも十分に平均的なプレイヤとしての成績が期待できるのでそれが実際どの程度の差があるか、各要素を比較することはどれだけ重要なのか、を数値的に明らかにしたいと考えている。



	% 本文4
\end{verbatim}
\end{itembox}

目次に続いて、論文のメイン、本文を記述する。アブストラクトと同様で、{\tt main.tex}に直接書くか、\verb|\include| コマンドを利用して別に用意したファイルを{\tt include}する。

本文の書き方は、第\ref{chap:latex}章で詳しく説明する。


\subsection{謝辞の出力}

\begin{itembox}[l]{{\tt main.tex}}
\begin{verbatim}
\begin{acknowledgment}

このテンプレートを改造するにあたって、@kurokoboとインターネット上のいくつかの修士論文などを参考にしました。感謝いたします。

\end{acknowledgment}
	% 謝辞。要独自コマンド、include先参照のこと
\end{verbatim}
\end{itembox}

本文のあとには、謝辞を出力する。\verb|begin{acknowledgment}| から \verb|end{acknowledgment}| の間に書いた文章が、謝辞として独立したページに出力される。アブストラクトや本文と同じで、{\tt main.tex}に直接書いてもよいし、\verb|\include| コマンドを利用して{\tt include}してもよい。


\subsection{参考文献の出力}

\begin{itembox}[l]{{\tt main.tex}}
\begin{verbatim}

\begin{bib}[100]
% BibTeXを使う場合
\bibliography{bib/main}

%\begin{thebibliography}{#1}
%
%  \bibitem{参照用名称}
%    著者名:
%    \newblock 文献名,
%    \newblock 書誌情報,出版年.
%
% \bibitem{hoge09}
%   ほげ山太郎,ほげ山次郎:
%   \newblock ほげほげ理論のHCI分野への応用,
%   \newblock ほげほげ学会論文誌,Vol.31,No.3,pp.194-201,2009.
%
% \bibitem{hoge08}
%   Taro Hogeyama, Jiro Hogeyama:
%   \newblock The Theory of Hoge,
%   \newblock {\it The Proceedings of The Hoge Society}, 2008.
%
%\end{thebibliography}

\end{bib}
	% 参考文献。要独自コマンド、include先参照のこと
\end{verbatim}
\end{itembox}

謝辞に続いて、参考文献を出力する。

参考文献リストは、\verb|\begin{bib}| から \verb|\end{bib}| の間に、\verb|\bibitem| コマンドを使って書く。

BibTeXを使う場合は、以下のようにする。

\begin{itembox}[l]{{\tt 91\_bibliography.tex}}
\begin{verbatim}
\begin{bib}[100]
\bibliography{main}
\end{bib}
\end{verbatim}
\end{itembox}

こうすると、\verb|main.bib|から使用した参考文献のみを抽出して出力してくれる。\verb|main.bib|の中身は以下のようになっていて、気の利いた論文検索サイトであればBibTeXをコピペできるようになっているので簡単に作れるはず。


\begin{itembox}[l]{{\tt 91\_bibliography.tex}}
\begin{verbatim}
@article{hoge09,
    author  = "ほげ山太郎 and ほげ山次郎",
    yomi    = "ほげやまたろう",
    title   = "ほげほげ理論のHCI分野への応用",
    journal = "ほげほげ学会論文誌",
    volume  = "31",
    number  = "3",
    pages   = "194-201",
    year    = "2009",
}
@inproceedings{hoge08,
    author     = "Taro Hogeyama and Jiro Hogeyama",
    title      = "The Theory of Hoge",
    booktitle  = "The Proceedings of The Hoge Society",
    year       = "2008"
}
\end{verbatim}
\end{itembox}


以下は、BibTeXを使わないで手で書く例。

\begin{itembox}[l]{{\tt 91\_bibliography.tex}}
\begin{verbatim}
@article{hoge09,
    author  = "ほげ山太郎 and ほげ山次郎",
    yomi    = "ほげやまたろう",
    title   = "ほげほげ理論のHCI分野への応用",
    journal = "ほげほげ学会論文誌",
    volume  = "31",
    number  = "3",
    pages   = "194-201",
    year    = "2009",
}
@inproceedings{hoge08,
    author     = "Taro Hogeyama and Jiro Hogeyama",
    title      = "The Theory of Hoge",
    booktitle  = "The Proceedings of The Hoge Society",
    year       = "2008"
}
\end{verbatim}
\end{itembox}


英語の文献の場合、慣例的に書誌名をイタリック体にすることが多いらしい。

\begin{itembox}[l]{{\tt 91\_bibliography.tex}}
\begin{verbatim}
\begin{bib}[100]
\begin{thebibliography}{#1}
% \bibitem{参照用名称}
%   著者名: 
%   \newblock 文献名,
%   \newblock 書誌情報,出版年.

\bibitem{hoge09}
  ほげ山太郎,ほげ山次郎:
  \newblock ほげほげ理論のHCI分野への応用,
  \newblock ほげほげ学会論文誌,Vol.31,No.3,pp.194-201,2009.

\bibitem{hoge08}
  Taro Hogeyama, Jiro Hogeyama:
  \newblock The Theory of Hoge,
  \newblock {\it The Proceedings of The Hoge Society}, 2008.
\end{thebibliography}
\end{bib}
\end{verbatim}
\end{itembox}

\verb|\bibitem| コマンド中、参照用名称は、本文から参考文献を参照するときに使うので、忘れずに書いておく。参照文献を本文中に参照するときには、\verb|\cite{参照用名称}| のように書けばよい。例えば、この文の末尾には \verb|\cite{hoge09}| と書いてあるので、自動で対応する番号が振られる\cite{hoge09}\cite{hoge08}。

参考文献リストの番号付けと、本文で参照したときの番号の挿入は、全部が自動で行われる。ただしこれも、第\ref{sec:toc}節で説明した目次の出力と同じで、一時ファイルを生成してからの挿入なので、正しく出力するには最低でも二回のコンパイルが必要。BibTeXを使用する場合は、\verb|platex|コマンドのあと\verb|pbibtex|コマンドを実行し、さらに2回\verb|platex|コマンドを実行するといいらしい。



\subsection{付録の出力}

\begin{itembox}[l]{{\tt main.tex}}
\begin{verbatim}
\appendix
\chapter{付録の例}

付録を無理矢理出力させるため、てきとうなことを書く。

\section{ほげ}

コマンドは本文と一緒。

\subsection{ふー}

本文と一緒。

\section{ほげほげ}

本文と一緒。

\subsection{ふーふー}

本文と一緒。
		% 付録
\end{verbatim}
\end{itembox}

必要であれば、論文の最後には付録を出力する。

\verb|\appendix| コマンド以降に書いたものは、すべて付録として扱われる。付録部分の書き方は通常の本文とまったく同じで、\verb|\appendix| コマンド以降に書くだけで勝手に付録用の体裁で出力される。
